% Options for packages loaded elsewhere
\PassOptionsToPackage{unicode}{hyperref}
\PassOptionsToPackage{hyphens}{url}
%
\documentclass[
]{article}
\usepackage{amsmath,amssymb}
\usepackage{iftex}
\ifPDFTeX
  \usepackage[T1]{fontenc}
  \usepackage[utf8]{inputenc}
  \usepackage{textcomp} % provide euro and other symbols
\else % if luatex or xetex
  \usepackage{unicode-math} % this also loads fontspec
  \defaultfontfeatures{Scale=MatchLowercase}
  \defaultfontfeatures[\rmfamily]{Ligatures=TeX,Scale=1}
\fi
\usepackage{lmodern}
\ifPDFTeX\else
  % xetex/luatex font selection
\fi
% Use upquote if available, for straight quotes in verbatim environments
\IfFileExists{upquote.sty}{\usepackage{upquote}}{}
\IfFileExists{microtype.sty}{% use microtype if available
  \usepackage[]{microtype}
  \UseMicrotypeSet[protrusion]{basicmath} % disable protrusion for tt fonts
}{}
\makeatletter
\@ifundefined{KOMAClassName}{% if non-KOMA class
  \IfFileExists{parskip.sty}{%
    \usepackage{parskip}
  }{% else
    \setlength{\parindent}{0pt}
    \setlength{\parskip}{6pt plus 2pt minus 1pt}}
}{% if KOMA class
  \KOMAoptions{parskip=half}}
\makeatother
\usepackage{xcolor}
\usepackage[margin=1in]{geometry}
\usepackage{color}
\usepackage{fancyvrb}
\newcommand{\VerbBar}{|}
\newcommand{\VERB}{\Verb[commandchars=\\\{\}]}
\DefineVerbatimEnvironment{Highlighting}{Verbatim}{commandchars=\\\{\}}
% Add ',fontsize=\small' for more characters per line
\usepackage{framed}
\definecolor{shadecolor}{RGB}{248,248,248}
\newenvironment{Shaded}{\begin{snugshade}}{\end{snugshade}}
\newcommand{\AlertTok}[1]{\textcolor[rgb]{0.94,0.16,0.16}{#1}}
\newcommand{\AnnotationTok}[1]{\textcolor[rgb]{0.56,0.35,0.01}{\textbf{\textit{#1}}}}
\newcommand{\AttributeTok}[1]{\textcolor[rgb]{0.13,0.29,0.53}{#1}}
\newcommand{\BaseNTok}[1]{\textcolor[rgb]{0.00,0.00,0.81}{#1}}
\newcommand{\BuiltInTok}[1]{#1}
\newcommand{\CharTok}[1]{\textcolor[rgb]{0.31,0.60,0.02}{#1}}
\newcommand{\CommentTok}[1]{\textcolor[rgb]{0.56,0.35,0.01}{\textit{#1}}}
\newcommand{\CommentVarTok}[1]{\textcolor[rgb]{0.56,0.35,0.01}{\textbf{\textit{#1}}}}
\newcommand{\ConstantTok}[1]{\textcolor[rgb]{0.56,0.35,0.01}{#1}}
\newcommand{\ControlFlowTok}[1]{\textcolor[rgb]{0.13,0.29,0.53}{\textbf{#1}}}
\newcommand{\DataTypeTok}[1]{\textcolor[rgb]{0.13,0.29,0.53}{#1}}
\newcommand{\DecValTok}[1]{\textcolor[rgb]{0.00,0.00,0.81}{#1}}
\newcommand{\DocumentationTok}[1]{\textcolor[rgb]{0.56,0.35,0.01}{\textbf{\textit{#1}}}}
\newcommand{\ErrorTok}[1]{\textcolor[rgb]{0.64,0.00,0.00}{\textbf{#1}}}
\newcommand{\ExtensionTok}[1]{#1}
\newcommand{\FloatTok}[1]{\textcolor[rgb]{0.00,0.00,0.81}{#1}}
\newcommand{\FunctionTok}[1]{\textcolor[rgb]{0.13,0.29,0.53}{\textbf{#1}}}
\newcommand{\ImportTok}[1]{#1}
\newcommand{\InformationTok}[1]{\textcolor[rgb]{0.56,0.35,0.01}{\textbf{\textit{#1}}}}
\newcommand{\KeywordTok}[1]{\textcolor[rgb]{0.13,0.29,0.53}{\textbf{#1}}}
\newcommand{\NormalTok}[1]{#1}
\newcommand{\OperatorTok}[1]{\textcolor[rgb]{0.81,0.36,0.00}{\textbf{#1}}}
\newcommand{\OtherTok}[1]{\textcolor[rgb]{0.56,0.35,0.01}{#1}}
\newcommand{\PreprocessorTok}[1]{\textcolor[rgb]{0.56,0.35,0.01}{\textit{#1}}}
\newcommand{\RegionMarkerTok}[1]{#1}
\newcommand{\SpecialCharTok}[1]{\textcolor[rgb]{0.81,0.36,0.00}{\textbf{#1}}}
\newcommand{\SpecialStringTok}[1]{\textcolor[rgb]{0.31,0.60,0.02}{#1}}
\newcommand{\StringTok}[1]{\textcolor[rgb]{0.31,0.60,0.02}{#1}}
\newcommand{\VariableTok}[1]{\textcolor[rgb]{0.00,0.00,0.00}{#1}}
\newcommand{\VerbatimStringTok}[1]{\textcolor[rgb]{0.31,0.60,0.02}{#1}}
\newcommand{\WarningTok}[1]{\textcolor[rgb]{0.56,0.35,0.01}{\textbf{\textit{#1}}}}
\usepackage{longtable,booktabs,array}
\usepackage{calc} % for calculating minipage widths
% Correct order of tables after \paragraph or \subparagraph
\usepackage{etoolbox}
\makeatletter
\patchcmd\longtable{\par}{\if@noskipsec\mbox{}\fi\par}{}{}
\makeatother
% Allow footnotes in longtable head/foot
\IfFileExists{footnotehyper.sty}{\usepackage{footnotehyper}}{\usepackage{footnote}}
\makesavenoteenv{longtable}
\usepackage{graphicx}
\makeatletter
\def\maxwidth{\ifdim\Gin@nat@width>\linewidth\linewidth\else\Gin@nat@width\fi}
\def\maxheight{\ifdim\Gin@nat@height>\textheight\textheight\else\Gin@nat@height\fi}
\makeatother
% Scale images if necessary, so that they will not overflow the page
% margins by default, and it is still possible to overwrite the defaults
% using explicit options in \includegraphics[width, height, ...]{}
\setkeys{Gin}{width=\maxwidth,height=\maxheight,keepaspectratio}
% Set default figure placement to htbp
\makeatletter
\def\fps@figure{htbp}
\makeatother
\setlength{\emergencystretch}{3em} % prevent overfull lines
\providecommand{\tightlist}{%
  \setlength{\itemsep}{0pt}\setlength{\parskip}{0pt}}
\setcounter{secnumdepth}{-\maxdimen} % remove section numbering
\usepackage{booktabs}
\usepackage{longtable}
\usepackage{array}
\usepackage{multirow}
\usepackage{wrapfig}
\usepackage{float}
\usepackage{colortbl}
\usepackage{pdflscape}
\usepackage{tabu}
\usepackage{threeparttable}
\usepackage{threeparttablex}
\usepackage[normalem]{ulem}
\usepackage{makecell}
\usepackage{xcolor}
\ifLuaTeX
  \usepackage{selnolig}  % disable illegal ligatures
\fi
\usepackage{bookmark}
\IfFileExists{xurl.sty}{\usepackage{xurl}}{} % add URL line breaks if available
\urlstyle{same}
\hypersetup{
  pdftitle={Wine Evaluation},
  pdfauthor={Shaya Engelman},
  hidelinks,
  pdfcreator={LaTeX via pandoc}}

\title{Wine Evaluation}
\author{Shaya Engelman}
\date{2024-12-18}

\begin{document}
\maketitle

\section{Introduction}\label{introduction}

A data set containing information on approximately 12,000 commercially
available wines and their variables mostly related to chemical
properties is analyzed for impact on sales and used to predict on sales
to give accurate forecasts for manufacturing. The response variable is
the number of sample cases of wine that were purchased by wine
distribution companies after sampling a wine. These cases would be used
to provide tasting samples to restaurants and wine stores around the
United States. The more sample cases purchased, the more likely is a
wine to be sold at a high end restaurant.

\subsection{Required Libraries}\label{required-libraries}

\begin{Shaded}
\begin{Highlighting}[]
\FunctionTok{library}\NormalTok{(tidyverse)}
\FunctionTok{library}\NormalTok{(janitor)}
\FunctionTok{library}\NormalTok{(knitr)}
\FunctionTok{library}\NormalTok{(kableExtra)}
\FunctionTok{library}\NormalTok{(latex2exp)}
\FunctionTok{library}\NormalTok{(psych)}
\FunctionTok{library}\NormalTok{(scales)}
\FunctionTok{library}\NormalTok{(stringr)}
\FunctionTok{library}\NormalTok{(ggcorrplot)}
\FunctionTok{library}\NormalTok{(ggmice)}
\FunctionTok{library}\NormalTok{(caret)}
\FunctionTok{library}\NormalTok{(mice)}
\FunctionTok{library}\NormalTok{(bestNormalize)}
\FunctionTok{library}\NormalTok{(e1071)}
\FunctionTok{library}\NormalTok{(diptest)}
\FunctionTok{library}\NormalTok{(MASS)}
\FunctionTok{library}\NormalTok{(performance)}
\end{Highlighting}
\end{Shaded}

\section{Data Exploration}\label{data-exploration}

To-Do List: 1. Check for typo's - No Typo's, all data is int 2. Check
for missing 3. Show distributions 4. Determine Categorical/Continuous

\subsection{Data Summary}\label{data-summary}

A table below expands on the variables included in analysis with
comments from domain experts on expected effects.

\newline

A quick look at the variables 5 number summary reveals that several
variables have large ranges which when relating to their mean may
suggest significantly different scales between variables, a high amount
of skew, bi-modal distributions, or outliers. FixedAcidity,
ResidualSugar, FreeSulfurDioxide, and TotalSulfurDioxide have fairly
extreme ranges in comparison to their means. Variables with Kurtosis
greater than 4 will have observations distributed into heavy or long
tails and may suggest numerous outliers, less than 2 suggest
distributions centered around their mean with short or thin tails. Many
of the variables are just below 2 suggesting many will have sharp peaks
around the mean. Only AcidIndex shows as a non-ordinal or discrete
distribution with extreme values of kurtosis, suggesting it will contain
many outliers.

\newpage

\textbf{Histograms}

As kurtosis foreshadowed, many of the distributions have sharp peaks at
the mean with only the AcidIndex showing a bi-modal distribution. With
the sharp centers around the peaks in the histograms, a high number of
outliers may present themselves.

\begin{center}\includegraphics{test_files/figure-latex/histograms-1} \end{center}

\newpage

\textbf{Bar Plots} There is a relatively normal distribution to
LabelAppeal, but both STARS and TARGET tend to favor their lower values
suggesting it's quite difficult to gain either a critic's praise or a
significant amount of cases sold.

\begin{center}\includegraphics{test_files/figure-latex/barplots-1} \end{center}

\textbf{Box Plots}

Boxplots reveal a significant number of residuals in all of the
variables. \newline

\begin{center}\includegraphics{test_files/figure-latex/boxplot-1} \end{center}

\textbf{Correlation Matrix}

The correlation matrix reveals a moderate relationship between STARS and
LabelAppeal with Target. Although both STARS and LabelAppeal seem to be
somewhat correlated to each other suggesting potential colinearity. The
AcidIndex, being a propietary method that aggregates across Acid
metrics, does show some relationship with FixedAcidity but is relatively
minor.

\includegraphics{test_files/figure-latex/corr-plot-1.pdf}

\textbf{Missing Values}

While missing values may be indicative of the target, the STARS variable
is missing 26\% of its values. Determining the relationship it has to
cases sold may be useful before removing it from the dataset. To view
the relationship of the ``missing'' STARS ratings, NA's have been
replaced with a category of ``Unrated'' and the bar plots are shown
again. Chlorides, FreeSulfurDioxide, Alcohol, and TotalSulfurDioxide are
missing around 5\% or about 600 values. Sulphates is missing about 10\%
of its values and about 1200 values.
\includegraphics{test_files/figure-latex/percent missing check-1.pdf}

\begin{Shaded}
\begin{Highlighting}[]
\FunctionTok{plot\_pattern}\NormalTok{(train, }\AttributeTok{square =} \ConstantTok{TRUE}\NormalTok{, }\AttributeTok{rotate =} \ConstantTok{TRUE}\NormalTok{, }\AttributeTok{npat =} \DecValTok{6}\NormalTok{)}
\end{Highlighting}
\end{Shaded}

\includegraphics{test_files/figure-latex/missing pattern-1.pdf}

While unrated wines typically aren't purchased, there are some that sell
about 3 cases. This might suggest that non-rated wines are not submitted
for critic's appraisal and should be used as a feature in the modeling.
This plot also reveals a heavy preference for 2 star wines.

\includegraphics{test_files/figure-latex/unnamed-chunk-3-1.pdf}

\includegraphics{test_files/figure-latex/unnamed-chunk-4-1.pdf}

The majority of 0 value appeals center on 4 cases sold and does tend to
show a linear relationship between the two. \newline

\includegraphics{test_files/figure-latex/unnamed-chunk-5-1.pdf}

\section{Data Preparation}\label{data-preparation}

Now that we have explored our data set, we can move on to data
preparation to prep out data for modeling and analysis.

\subsection{Data Wrangling}\label{data-wrangling}

To do list: 1. Split the data into training and testing sets 2. Impute
missing values 3. Normalize the data 4. Deal with outliers 5. One hot
encode the categorical variables

The data has already been partially cleaned with the removal of the
INDEX variable. The missing values in STARS were replaced with
``Unrated'' to indicate non-rated wines.

\subsubsection{Data Imputation}\label{data-imputation}

Before we can impute missing values, we perform the train-test split to
avoid data leakage:

Now, we can impute the missing values in the training and testing data
sets. We will use the MICE package to impute the missing values. In the
imputation process, we will exclude the TARGET variable from the
predictors, as the target variable should not be used to predict the
missing values of the predictors. All imputation will be done for all
three of the training, testing, and evaluation data sets. In order to
make the dataframes match we drop the INDEX column from the evaluation
data set.

Now that we've imputed the missing values, we can compare the summary
statistics of the original data and the imputed data. The summary
statistics are calculated for the following variables: Chlorides,
FreeSulfurDioxide, Alcohol, TotalSulfurDioxide, pH, and Sulphates. The
summary statistics are calculated for the full training data set, the
training data set after imputation, and the testing data set after
imputation. The summary statistics are calculated for the minimum, 1st
quartile, median, mean, 3rd quartile, and maximum values of the
variables. The summary statistics are then compared across the three
data sets to see how the imputation process has affected the data.

The above table is very encouraging. The summary statistics for the
variables with missing data did not seem to change much after the
imputation. Most of the discrepancies appear in the test data set, this
plausibly due to the smaller sample size.

\subsubsection{Transformations}\label{transformations}

Again, we must be quite careful to avoid data leakage, calculating the
parameters for these transformations using only the training set, and
then applying the transformations to our other sets using the same
parameters.

The transformations almost completely got rid of any skew in our data.
We can visualize this using by recreating the histograms with the
transformed data.

\includegraphics{test_files/figure-latex/histograms_2-1.pdf}

While the first table seems to suggest that AcidIndex\_transformed had
its skewness lowered to a relatively insignificant amount, the histogram
reveals that the variable still seemingly is bimodal. This may suggest
that the transformation was not the best choice for this variable and
grouping the data may be a better choice. However, upon further
investigation, the appearance of bimodality may be due to the amount of
bins selected for the histograms. More bins reveal a more normal
distribution. We can test whether it is bimodal using a dip test.

\begin{verbatim}
## 
##  Hartigans' dip test for unimodality / multimodality
## 
## data:  train_data_transformed$AcidIndex
## D = 0.15872, p-value < 2.2e-16
## alternative hypothesis: non-unimodal, i.e., at least bimodal
\end{verbatim}

The extremely low p-value suggests that the AcidIndex variable is
bimodal. We will group the data into two categories to deal with this
issue.

\includegraphics{test_files/figure-latex/unnamed-chunk-9-1.pdf}

The resulting groups are shown in the histogram above. The plot reveals
that, while not evenly distributed, there really is only one group. The
appearance of bimodality is likely due to the much larger amount of the
non-median group. We will not group the data and move on to dealing with
outliers.

Now that we've transformed our data, we can move on to dealing with
outliers.

\subsubsection{Outliers}\label{outliers}

We will use the IQR method to detect outliers in the data. The IQR
method is a robust method for detecting outliers that is not sensitive
to the presence of extreme values. The IQR method defines an outlier as
any value that is below Q1 - 1.5 * IQR or above Q3 + 1.5 * IQR. The
lower and upper limits for each variable are calculated using the IQR
method.

Using the IQR limits, there is a significant amount of outliers in the
data. The transformation process did not impact the number of outliers
in the data. Using a Box-Cox transformation might have been a better way
to get rid of the outliers but it was not an option for many of the
variables due to them containing negative and zero values.

Ultimately, due to the large amount of outliers, removing them would
result in a significant loss of data. We will keep the outliers in the
data and move on to one-hot encoding the categorical variables.

\subsubsection{One-Hot Encoding}\label{one-hot-encoding}

We have two factor columns in LabelAppeal and STARS. LabelAppeal can be
converted to numeric as it is ordinal. While STARS is also ordinal, it
also has an `unrated' category. We will one-hot encode this column but
also keep the original column for now.

After our data has been prepped, we can now move on to modeling.

\section{Modeling}\label{modeling}

With the data exploration and preparation out of the way, we turn to
build different types of regression models to predict the number of
cases of wine ordered by distributors. Again, the response variable is
the \emph{count} of cases, and so it is appropriate to consider Poisson
regression, negative binomial regression, and multiple linear
regression. We build models of each type with some commentary, and then
we will consider more generally how the models compare to one another.

\subsection{Poisson Regression}\label{poisson-regression}

We first consider Poisson regression models. Now, it's critical to note,
despite the transformations we performed in the previous section,
Poisson models do not require normally distributed data, and so
leveraging transformed data is actually counter-productive. As such, the
relevant dataframes are:

\begin{itemize}
\tightlist
\item
  train\_data\_imputed
\item
  test\_data\_imputed
\item
  eval\_data\_imputed
\end{itemize}

\subsubsection{Poisson Model 1}\label{poisson-model-1}

We start with a rather simple model, with all the variables along with a
few more sophisticated variables:

\begin{enumerate}
\def\labelenumi{\arabic{enumi}.}
\tightlist
\item
  Alcohol:LabelAppeal in case these two variables have an especially
  strong combined effect
\item
  STARS:Alcohol since high quality wines with certain alcohol content
  might sell especially well
\item
  LabelAppeal:STARS in case people might be especially likely to buy
  visually appealing and highly rated wines
\item
  Alcohol\^{}2 as intuitively alcohol content doesn't have a strictly
  linear relationship with the target variable
\end{enumerate}

\begin{verbatim}
##  Factor w/ 9 levels "0","1","2","3",..: 4 4 6 4 5 1 5 7 1 5 ...
\end{verbatim}

\begin{verbatim}
## 
## Call:
## glm(formula = TARGET ~ Alcohol + Alcohol^2 + LabelAppeal + STARS + 
##     AcidIndex + Chlorides + CitricAcid + Density + FixedAcidity + 
##     FreeSulfurDioxide + ResidualSugar + Sulphates + TotalSulfurDioxide + 
##     VolatileAcidity + pH + Alcohol:LabelAppeal + STARS:Alcohol + 
##     LabelAppeal:STARS, family = poisson(), data = train_data_imputed)
## 
## Coefficients: (1 not defined because of singularities)
##                              Estimate Std. Error z value Pr(>|z|)    
## (Intercept)                 1.344e+00  1.233e-01  10.893  < 2e-16 ***
## Alcohol                     8.200e-03  5.323e-03   1.541 0.123437    
## LabelAppeal-1               3.952e-01  6.877e-02   5.747 9.10e-09 ***
## LabelAppeal0                6.420e-01  6.722e-02   9.551  < 2e-16 ***
## LabelAppeal1                7.300e-01  6.904e-02  10.573  < 2e-16 ***
## LabelAppeal2                7.548e-01  8.778e-02   8.599  < 2e-16 ***
## STARS2                      2.600e-01  5.629e-02   4.618 3.87e-06 ***
## STARS3                      5.118e-01  8.568e-02   5.973 2.33e-09 ***
## STARS4                      8.040e-01  6.222e-02  12.922  < 2e-16 ***
## STARSUnrated               -5.029e-01  5.674e-02  -8.862  < 2e-16 ***
## AcidIndex                  -7.576e-02  2.453e-03 -30.881  < 2e-16 ***
## Chlorides                  -3.808e-02  8.609e-03  -4.423 9.71e-06 ***
## CitricAcid                  5.293e-03  3.182e-03   1.663 0.096234 .  
## Density                    -3.336e-01  1.029e-01  -3.243 0.001183 ** 
## FixedAcidity                1.147e-04  4.369e-04   0.263 0.792929    
## FreeSulfurDioxide           9.423e-05  1.841e-05   5.118 3.09e-07 ***
## ResidualSugar               1.471e-04  8.122e-05   1.811 0.070104 .  
## Sulphates                  -9.047e-03  2.953e-03  -3.063 0.002189 ** 
## TotalSulfurDioxide          8.507e-05  1.180e-05   7.208 5.66e-13 ***
## VolatileAcidity            -3.508e-02  3.466e-03 -10.122  < 2e-16 ***
## pH                         -1.512e-02  3.989e-03  -3.791 0.000150 ***
## Alcohol:LabelAppeal-1      -4.535e-03  5.507e-03  -0.823 0.410277    
## Alcohol:LabelAppeal0       -8.650e-03  5.367e-03  -1.612 0.107037    
## Alcohol:LabelAppeal1       -6.552e-03  5.481e-03  -1.195 0.232001    
## Alcohol:LabelAppeal2       -2.205e-02  6.358e-03  -3.468 0.000524 ***
## Alcohol:STARS2              7.646e-03  2.031e-03   3.764 0.000167 ***
## Alcohol:STARS3              3.722e-03  2.234e-03   1.666 0.095685 .  
## Alcohol:STARS4              3.878e-03  3.233e-03   1.199 0.230435    
## Alcohol:STARSUnrated        1.849e-05  2.792e-03   0.007 0.994716    
## LabelAppeal-1:STARS2       -9.189e-02  5.377e-02  -1.709 0.087490 .  
## LabelAppeal0:STARS2        -4.243e-02  5.275e-02  -0.804 0.421141    
## LabelAppeal1:STARS2         5.258e-02  5.419e-02   0.970 0.331901    
## LabelAppeal2:STARS2         3.508e-01  7.237e-02   4.847 1.25e-06 ***
## LabelAppeal-1:STARS3       -1.215e-01  8.480e-02  -1.433 0.151866    
## LabelAppeal0:STARS3        -1.384e-01  8.315e-02  -1.665 0.095960 .  
## LabelAppeal1:STARS3        -8.809e-02  8.397e-02  -1.049 0.294134    
## LabelAppeal2:STARS3         2.314e-01  9.602e-02   2.410 0.015964 *  
## LabelAppeal-1:STARS4       -1.855e-01  7.662e-02  -2.421 0.015470 *  
## LabelAppeal0:STARS4        -2.962e-01  5.521e-02  -5.366 8.06e-08 ***
## LabelAppeal1:STARS4        -2.817e-01  5.527e-02  -5.098 3.44e-07 ***
## LabelAppeal2:STARS4                NA         NA      NA       NA    
## LabelAppeal-1:STARSUnrated -9.569e-02  5.166e-02  -1.852 0.064001 .  
## LabelAppeal0:STARSUnrated  -2.357e-01  5.051e-02  -4.667 3.05e-06 ***
## LabelAppeal1:STARSUnrated  -5.390e-01  5.437e-02  -9.912  < 2e-16 ***
## LabelAppeal2:STARSUnrated  -6.257e-02  7.974e-02  -0.785 0.432609    
## ---
## Signif. codes:  0 '***' 0.001 '**' 0.01 '*' 0.05 '.' 0.1 ' ' 1
## 
## (Dispersion parameter for poisson family taken to be 1)
## 
##     Null deviance: 80043  on 44794  degrees of freedom
## Residual deviance: 47535  on 44751  degrees of freedom
## AIC: 159458
## 
## Number of Fisher Scoring iterations: 6
\end{verbatim}

The model seems promising; for example there's major reduction in
deviance from the null model to the full model. Still, there's much work
to be done. We start by noting there are undefined coefficients because
of singularities. Let's take a look at potential multicollinearity.

\begin{verbatim}
## Model :
## TARGET ~ Alcohol + Alcohol^2 + LabelAppeal + STARS + AcidIndex + 
##     Chlorides + CitricAcid + Density + FixedAcidity + FreeSulfurDioxide + 
##     ResidualSugar + Sulphates + TotalSulfurDioxide + VolatileAcidity + 
##     pH + Alcohol:LabelAppeal + STARS:Alcohol + LabelAppeal:STARS
## 
## Complete :
##                     (Intercept) Alcohol LabelAppeal-1 LabelAppeal0 LabelAppeal1
## LabelAppeal2:STARS4  0           0       0             0            0          
##                     LabelAppeal2 STARS2 STARS3 STARS4 STARSUnrated AcidIndex
## LabelAppeal2:STARS4  0            0      0      1      0            0       
##                     Chlorides CitricAcid Density FixedAcidity FreeSulfurDioxide
## LabelAppeal2:STARS4  0         0          0       0            0               
##                     ResidualSugar Sulphates TotalSulfurDioxide VolatileAcidity
## LabelAppeal2:STARS4  0             0         0                  0             
##                     pH Alcohol:LabelAppeal-1 Alcohol:LabelAppeal0
## LabelAppeal2:STARS4  0  0                     0                  
##                     Alcohol:LabelAppeal1 Alcohol:LabelAppeal2 Alcohol:STARS2
## LabelAppeal2:STARS4  0                    0                    0            
##                     Alcohol:STARS3 Alcohol:STARS4 Alcohol:STARSUnrated
## LabelAppeal2:STARS4  0              0              0                  
##                     LabelAppeal-1:STARS2 LabelAppeal0:STARS2
## LabelAppeal2:STARS4  0                    0                 
##                     LabelAppeal1:STARS2 LabelAppeal2:STARS2
## LabelAppeal2:STARS4  0                   0                 
##                     LabelAppeal-1:STARS3 LabelAppeal0:STARS3
## LabelAppeal2:STARS4  0                    0                 
##                     LabelAppeal1:STARS3 LabelAppeal2:STARS3
## LabelAppeal2:STARS4  0                   0                 
##                     LabelAppeal-1:STARS4 LabelAppeal0:STARS4
## LabelAppeal2:STARS4 -1                   -1                 
##                     LabelAppeal1:STARS4 LabelAppeal-1:STARSUnrated
## LabelAppeal2:STARS4 -1                   0                        
##                     LabelAppeal0:STARSUnrated LabelAppeal1:STARSUnrated
## LabelAppeal2:STARS4  0                         0                       
##                     LabelAppeal2:STARSUnrated
## LabelAppeal2:STARS4  0
\end{verbatim}

\begin{verbatim}
## 
## Call:
## glm(formula = TARGET ~ Alcohol + I(Alcohol^2) + LabelAppeal + 
##     STARS + AcidIndex + Chlorides + CitricAcid + Density + FixedAcidity + 
##     FreeSulfurDioxide + ResidualSugar + Sulphates + TotalSulfurDioxide + 
##     VolatileAcidity + pH + Alcohol:LabelAppeal + STARS:Alcohol, 
##     family = poisson(), data = train_data_imputed)
## 
## Coefficients:
##                         Estimate Std. Error z value Pr(>|z|)    
## (Intercept)            1.478e+00  1.211e-01  12.205  < 2e-16 ***
## Alcohol                1.814e-03  5.754e-03   0.315 0.752540    
## I(Alcohol^2)           2.115e-04  1.027e-04   2.059 0.039449 *  
## LabelAppeal-1          3.036e-01  6.323e-02   4.801 1.58e-06 ***
## LabelAppeal0           5.301e-01  6.170e-02   8.592  < 2e-16 ***
## LabelAppeal1           6.403e-01  6.282e-02  10.193  < 2e-16 ***
## LabelAppeal2           9.445e-01  7.224e-02  13.074  < 2e-16 ***
## STARS2                 2.554e-01  2.234e-02  11.432  < 2e-16 ***
## STARS3                 4.111e-01  2.498e-02  16.456  < 2e-16 ***
## STARS4                 5.284e-01  3.691e-02  14.319  < 2e-16 ***
## STARSUnrated          -7.541e-01  3.044e-02 -24.774  < 2e-16 ***
## AcidIndex             -7.803e-02  2.445e-03 -31.913  < 2e-16 ***
## Chlorides             -3.866e-02  8.609e-03  -4.491 7.09e-06 ***
## CitricAcid             6.276e-03  3.184e-03   1.971 0.048704 *  
## Density               -3.394e-01  1.028e-01  -3.302 0.000959 ***
## FixedAcidity           1.370e-04  4.367e-04   0.314 0.753636    
## FreeSulfurDioxide      9.861e-05  1.840e-05   5.359 8.38e-08 ***
## ResidualSugar          1.619e-04  8.111e-05   1.996 0.045970 *  
## Sulphates             -8.979e-03  2.951e-03  -3.043 0.002345 ** 
## TotalSulfurDioxide     8.624e-05  1.179e-05   7.318 2.52e-13 ***
## VolatileAcidity       -3.567e-02  3.462e-03 -10.304  < 2e-16 ***
## pH                    -1.513e-02  3.988e-03  -3.793 0.000149 ***
## Alcohol:LabelAppeal-1 -3.224e-03  5.469e-03  -0.590 0.555497    
## Alcohol:LabelAppeal0  -6.556e-03  5.327e-03  -1.231 0.218415    
## Alcohol:LabelAppeal1  -3.875e-03  5.435e-03  -0.713 0.475845    
## Alcohol:LabelAppeal2  -1.860e-02  6.303e-03  -2.951 0.003163 ** 
## Alcohol:STARS2         6.739e-03  2.015e-03   3.345 0.000823 ***
## Alcohol:STARS3         3.169e-03  2.222e-03   1.426 0.153860    
## Alcohol:STARS4         2.983e-03  3.217e-03   0.927 0.353707    
## Alcohol:STARSUnrated   9.260e-04  2.781e-03   0.333 0.739112    
## ---
## Signif. codes:  0 '***' 0.001 '**' 0.01 '*' 0.05 '.' 0.1 ' ' 1
## 
## (Dispersion parameter for poisson family taken to be 1)
## 
##     Null deviance: 80043  on 44794  degrees of freedom
## Residual deviance: 48067  on 44765  degrees of freedom
## AIC: 159962
## 
## Number of Fisher Scoring iterations: 6
\end{verbatim}

\begin{table}[H]
\centering\centering
\caption{\label{tab:unnamed-chunk-16}VIF Values simple model 2}
\centering
\begin{tabu} to \linewidth {>{\raggedright}X>{\raggedleft}X>{\raggedleft}X>{\raggedleft}X>{\raggedleft}X>{\raggedleft}X>{\raggedleft}X>{\raggedleft}X}
\hline
Term & VIF & VIF\_CI\_low & VIF\_CI\_high & SE\_factor & Tolerance & Tolerance\_CI\_low & Tolerance\_CI\_high\\
\hline
Alcohol & 62.616827 & 61.477857 & 63.777246 & 7.913080 & 0.0159701 & 0.0156796 & 0.0162660\\
\hline
I(Alcohol\textasciicircum{}2) & 9.975630 & 9.801962 & 10.152725 & 3.158422 & 0.1002443 & 0.0984957 & 0.1020204\\
\hline
LabelAppeal & 9648.902267 & 9471.976675 & 9829.132974 & 98.228826 & 0.0001036 & 0.0001017 & 0.0001056\\
\hline
STARS & 8808.211331 & 8646.701743 & 8972.738064 & 93.852071 & 0.0001135 & 0.0001114 & 0.0001157\\
\hline
AcidIndex & 1.063457 & 1.053840 & 1.074791 & 1.031241 & 0.9403296 & 0.9304135 & 0.9489102\\
\hline
Chlorides & 1.005769 & 1.001139 & 1.029226 & 1.002880 & 0.9942642 & 0.9716036 & 0.9988626\\
\hline
CitricAcid & 1.008438 & 1.002766 & 1.025737 & 1.004210 & 0.9916325 & 0.9749088 & 0.9972411\\
\hline
Density & 1.005458 & 1.000984 & 1.030295 & 1.002725 & 0.9945712 & 0.9705957 & 0.9990175\\
\hline
FixedAcidity & 1.024270 & 1.016272 & 1.036199 & 1.012062 & 0.9763054 & 0.9650659 & 0.9839888\\
\hline
FreeSulfurDioxide & 1.006183 & 1.001359 & 1.028133 & 1.003087 & 0.9938548 & 0.9726369 & 0.9986429\\
\hline
ResidualSugar & 1.003295 & 1.000195 & 1.055650 & 1.001646 & 0.9967154 & 0.9472837 & 0.9998049\\
\hline
Sulphates & 1.004457 & 1.000549 & 1.036204 & 1.002226 & 0.9955626 & 0.9650609 & 0.9994516\\
\hline
TotalSulfurDioxide & 1.005488 & 1.000998 & 1.030182 & 1.002740 & 0.9945418 & 0.9707026 & 0.9990031\\
\hline
VolatileAcidity & 1.007462 & 1.002119 & 1.026271 & 1.003724 & 0.9925937 & 0.9744016 & 0.9978852\\
\hline
pH & 1.008664 & 1.002923 & 1.025681 & 1.004323 & 0.9914106 & 0.9749619 & 0.9970857\\
\hline
Alcohol:LabelAppeal & 61462.048045 & 60335.009097 & 62610.140075 & 247.915405 & 0.0000163 & 0.0000160 & 0.0000166\\
\hline
Alcohol:STARS & 13147.238693 & 12906.163046 & 13392.817768 & 114.661409 & 0.0000761 & 0.0000747 & 0.0000775\\
\hline
\end{tabu}
\end{table}

There's clearly a high degree of collinearity, but it's critical to
remove one column at a time and reassess colinearity:

\begin{longtable}[]{@{}
  >{\raggedright\arraybackslash}p{(\columnwidth - 14\tabcolsep) * \real{0.1667}}
  >{\raggedleft\arraybackslash}p{(\columnwidth - 14\tabcolsep) * \real{0.1140}}
  >{\raggedleft\arraybackslash}p{(\columnwidth - 14\tabcolsep) * \real{0.1140}}
  >{\raggedleft\arraybackslash}p{(\columnwidth - 14\tabcolsep) * \real{0.1140}}
  >{\raggedleft\arraybackslash}p{(\columnwidth - 14\tabcolsep) * \real{0.0965}}
  >{\raggedleft\arraybackslash}p{(\columnwidth - 14\tabcolsep) * \real{0.0877}}
  >{\raggedleft\arraybackslash}p{(\columnwidth - 14\tabcolsep) * \real{0.1491}}
  >{\raggedleft\arraybackslash}p{(\columnwidth - 14\tabcolsep) * \real{0.1579}}@{}}
\toprule\noalign{}
\begin{minipage}[b]{\linewidth}\raggedright
Term
\end{minipage} & \begin{minipage}[b]{\linewidth}\raggedleft
VIF
\end{minipage} & \begin{minipage}[b]{\linewidth}\raggedleft
VIF\_CI\_low
\end{minipage} & \begin{minipage}[b]{\linewidth}\raggedleft
VIF\_CI\_high
\end{minipage} & \begin{minipage}[b]{\linewidth}\raggedleft
SE\_factor
\end{minipage} & \begin{minipage}[b]{\linewidth}\raggedleft
Tolerance
\end{minipage} & \begin{minipage}[b]{\linewidth}\raggedleft
Tolerance\_CI\_low
\end{minipage} & \begin{minipage}[b]{\linewidth}\raggedleft
Tolerance\_CI\_high
\end{minipage} \\
\midrule\noalign{}
\endhead
\bottomrule\noalign{}
\endlastfoot
Alcohol & 13.336455 & 13.101136 & 13.576349 & 3.651911 & 0.0749824 &
0.0736575 & 0.0763293 \\
I(Alcohol\^{}2) & 9.951011 & 9.777779 & 10.127662 & 3.154522 & 0.1004923
& 0.0987395 & 0.1022727 \\
LabelAppeal & 1.138945 & 1.127608 & 1.151289 & 1.067214 & 0.8780054 &
0.8685913 & 0.8868329 \\
STARS & 7702.157091 & 7560.917052 & 7846.035882 & 87.761934 & 0.0001298
& 0.0001275 & 0.0001323 \\
AcidIndex & 1.062630 & 1.053036 & 1.073960 & 1.030839 & 0.9410613 &
0.9311336 & 0.9496352 \\
Chlorides & 1.005394 & 1.000952 & 1.030556 & 1.002693 & 0.9946352 &
0.9703504 & 0.9990488 \\
CitricAcid & 1.007554 & 1.002178 & 1.026198 & 1.003770 & 0.9925022 &
0.9744705 & 0.9978264 \\
Density & 1.005196 & 1.000859 & 1.031420 & 1.002595 & 0.9948305 &
0.9695372 & 0.9991413 \\
FixedAcidity & 1.024144 & 1.016154 & 1.036084 & 1.012000 & 0.9764256 &
0.9651730 & 0.9841024 \\
FreeSulfurDioxide & 1.005811 & 1.001160 & 1.029103 & 1.002901 &
0.9942224 & 0.9717200 & 0.9988410 \\
ResidualSugar & 1.003246 & 1.000184 & 1.057221 & 1.001622 & 0.9967644 &
0.9458759 & 0.9998159 \\
Sulphates & 1.003738 & 1.000309 & 1.045273 & 1.001867 & 0.9962755 &
0.9566876 & 0.9996914 \\
TotalSulfurDioxide & 1.004897 & 1.000727 & 1.033014 & 1.002446 &
0.9951265 & 0.9680415 & 0.9992740 \\
VolatileAcidity & 1.006717 & 1.001662 & 1.027138 & 1.003353 & 0.9933278
& 0.9735790 & 0.9983402 \\
pH & 1.008217 & 1.002615 & 1.025816 & 1.004100 & 0.9918499 & 0.9748339 &
0.9973914 \\
Alcohol:STARS & 11232.455188 & 11026.473271 & 11442.285333 & 105.983278
& 0.0000890 & 0.0000874 & 0.0000907 \\
\end{longtable}

\begin{table}[H]
\centering\centering
\caption{\label{tab:unnamed-chunk-18}VIF Values simple model 2}
\centering
\begin{tabu} to \linewidth {>{\raggedright}X>{\raggedleft}X>{\raggedleft}X>{\raggedleft}X>{\raggedleft}X>{\raggedleft}X>{\raggedleft}X>{\raggedleft}X}
\hline
Term & VIF & VIF\_CI\_low & VIF\_CI\_high & SE\_factor & Tolerance & Tolerance\_CI\_low & Tolerance\_CI\_high\\
\hline
Alcohol & 9.907412 & 9.734964 & 10.083264 & 3.147604 & 0.1009345 & 0.0991742 & 0.1027225\\
\hline
I(Alcohol\textasciicircum{}2) & 9.889156 & 9.717043 & 10.064668 & 3.144703 & 0.1011209 & 0.0993575 & 0.1029120\\
\hline
LabelAppeal & 1.137419 & 1.126113 & 1.149740 & 1.066499 & 0.8791833 & 0.8697620 & 0.8880108\\
\hline
STARS & 1.170559 & 1.158588 & 1.183435 & 1.081924 & 0.8542924 & 0.8449979 & 0.8631198\\
\hline
AcidIndex & 1.061721 & 1.052151 & 1.073046 & 1.030398 & 0.9418672 & 0.9319266 & 0.9504335\\
\hline
Chlorides & 1.004810 & 1.000689 & 1.033565 & 1.002402 & 0.9952134 & 0.9675249 & 0.9993113\\
\hline
CitricAcid & 1.007451 & 1.002112 & 1.026286 & 1.003719 & 0.9926038 & 0.9743877 & 0.9978922\\
\hline
Density & 1.004941 & 1.000745 & 1.032762 & 1.002468 & 0.9950833 & 0.9682775 & 0.9992554\\
\hline
FixedAcidity & 1.023935 & 1.015961 & 1.035893 & 1.011897 & 0.9766241 & 0.9653503 & 0.9842895\\
\hline
FreeSulfurDioxide & 1.005755 & 1.001131 & 1.029278 & 1.002873 & 0.9942781 & 0.9715549 & 0.9988701\\
\hline
ResidualSugar & 1.003118 & 1.000157 & 1.061810 & 1.001558 & 0.9968917 & 0.9417877 & 0.9998427\\
\hline
Sulphates & 1.003582 & 1.000265 & 1.048346 & 1.001789 & 0.9964310 & 0.9538838 & 0.9997347\\
\hline
TotalSulfurDioxide & 1.004041 & 1.000402 & 1.040660 & 1.002019 & 0.9959749 & 0.9609283 & 0.9995985\\
\hline
VolatileAcidity & 1.006550 & 1.001565 & 1.027412 & 1.003270 & 0.9934923 & 0.9733190 & 0.9984372\\
\hline
pH & 1.007897 & 1.002401 & 1.025972 & 1.003941 & 0.9921648 & 0.9746857 & 0.9976046\\
\hline
\end{tabu}
\end{table}

At this point, we've removed all high correlation variables. Now, there
are still two variables with fairly high VIFs, namely Alcohol and
I(Alcohol\^{}2)--this is unsurprising to say the least. It would be odd
to remove only the former term, so let's see the model summary and
consider whether we ought to remove I(Alcohol\^{}2):

\begin{verbatim}
## 
## Call:
## glm(formula = TARGET ~ Alcohol + I(Alcohol^2) + LabelAppeal + 
##     STARS + AcidIndex + Chlorides + CitricAcid + Density + FixedAcidity + 
##     FreeSulfurDioxide + ResidualSugar + Sulphates + TotalSulfurDioxide + 
##     VolatileAcidity + pH, family = poisson(), data = train_data_imputed)
## 
## Coefficients:
##                      Estimate Std. Error z value Pr(>|z|)    
## (Intercept)         1.505e+00  1.065e-01  14.134  < 2e-16 ***
## Alcohol            -4.225e-04  2.289e-03  -0.185 0.853539    
## I(Alcohol^2)        2.183e-04  1.023e-04   2.134 0.032804 *  
## LabelAppeal-1       2.688e-01  2.074e-02  12.962  < 2e-16 ***
## LabelAppeal0        4.599e-01  2.027e-02  22.689  < 2e-16 ***
## LabelAppeal1        5.983e-01  2.060e-02  29.041  < 2e-16 ***
## LabelAppeal2        7.446e-01  2.318e-02  32.127  < 2e-16 ***
## STARS2              3.258e-01  7.656e-03  42.553  < 2e-16 ***
## STARS3              4.437e-01  8.361e-03  53.075  < 2e-16 ***
## STARS4              5.586e-01  1.150e-02  48.565  < 2e-16 ***
## STARSUnrated       -7.445e-01  1.046e-02 -71.179  < 2e-16 ***
## AcidIndex          -7.794e-02  2.442e-03 -31.914  < 2e-16 ***
## Chlorides          -3.930e-02  8.603e-03  -4.568 4.91e-06 ***
## CitricAcid          6.049e-03  3.183e-03   1.900 0.057403 .  
## Density            -3.418e-01  1.027e-01  -3.328 0.000875 ***
## FixedAcidity        1.468e-04  4.365e-04   0.336 0.736670    
## FreeSulfurDioxide   9.913e-05  1.840e-05   5.387 7.16e-08 ***
## ResidualSugar       1.618e-04  8.110e-05   1.995 0.046018 *  
## Sulphates          -8.779e-03  2.950e-03  -2.976 0.002917 ** 
## TotalSulfurDioxide  8.684e-05  1.178e-05   7.374 1.66e-13 ***
## VolatileAcidity    -3.535e-02  3.460e-03 -10.218  < 2e-16 ***
## pH                 -1.541e-02  3.987e-03  -3.864 0.000112 ***
## ---
## Signif. codes:  0 '***' 0.001 '**' 0.01 '*' 0.05 '.' 0.1 ' ' 1
## 
## (Dispersion parameter for poisson family taken to be 1)
## 
##     Null deviance: 80043  on 44794  degrees of freedom
## Residual deviance: 48099  on 44773  degrees of freedom
## AIC: 159978
## 
## Number of Fisher Scoring iterations: 6
\end{verbatim}

Indeed, I(Alcohol\^{}2) \emph{is} statistically significant, so we won't
remove it. However, there are a couple variables that appear less
promising, and we will removes those one at a time (a manual backwards
elimination process).

\begin{verbatim}
## 
## Call:
## glm(formula = TARGET ~ Alcohol + I(Alcohol^2) + LabelAppeal + 
##     STARS + AcidIndex + Chlorides + Density + FreeSulfurDioxide + 
##     Sulphates + TotalSulfurDioxide + VolatileAcidity + pH, family = poisson(), 
##     data = train_data_imputed)
## 
## Coefficients:
##                      Estimate Std. Error z value Pr(>|z|)    
## (Intercept)         1.507e+00  1.064e-01  14.158  < 2e-16 ***
## Alcohol            -4.546e-04  2.289e-03  -0.199 0.842536    
## I(Alcohol^2)        2.199e-04  1.022e-04   2.151 0.031464 *  
## LabelAppeal-1       2.685e-01  2.073e-02  12.948  < 2e-16 ***
## LabelAppeal0        4.595e-01  2.027e-02  22.668  < 2e-16 ***
## LabelAppeal1        5.980e-01  2.060e-02  29.029  < 2e-16 ***
## LabelAppeal2        7.450e-01  2.317e-02  32.150  < 2e-16 ***
## STARS2              3.262e-01  7.654e-03  42.618  < 2e-16 ***
## STARS3              4.440e-01  8.360e-03  53.113  < 2e-16 ***
## STARS4              5.589e-01  1.150e-02  48.596  < 2e-16 ***
## STARSUnrated       -7.447e-01  1.046e-02 -71.201  < 2e-16 ***
## AcidIndex          -7.751e-02  2.412e-03 -32.128  < 2e-16 ***
## Chlorides          -3.975e-02  8.601e-03  -4.622 3.80e-06 ***
## Density            -3.439e-01  1.027e-01  -3.349 0.000811 ***
## FreeSulfurDioxide   9.995e-05  1.840e-05   5.432 5.56e-08 ***
## Sulphates          -8.900e-03  2.948e-03  -3.019 0.002539 ** 
## TotalSulfurDioxide  8.728e-05  1.177e-05   7.413 1.23e-13 ***
## VolatileAcidity    -3.551e-02  3.459e-03 -10.267  < 2e-16 ***
## pH                 -1.527e-02  3.986e-03  -3.830 0.000128 ***
## ---
## Signif. codes:  0 '***' 0.001 '**' 0.01 '*' 0.05 '.' 0.1 ' ' 1
## 
## (Dispersion parameter for poisson family taken to be 1)
## 
##     Null deviance: 80043  on 44794  degrees of freedom
## Residual deviance: 48107  on 44776  degrees of freedom
## AIC: 159980
## 
## Number of Fisher Scoring iterations: 6
\end{verbatim}

There are a number of takeaways from this model summary, most of which
are totally expected. First, only the quadratic term for alcohol is
significant; this suggests that after a certain point, small changes in
alcohol content--past a certain point--can have a large impact on the
target variable. Second, the higher the label appeal level, the higher
the log counts of cases ordered. Third, higher star ratings are highly
associated with higher values for the target variable; being unrated
significantly decreases the log count of cases ordered--we return to
this point momentarily. Finally, a number of the chemical properties
have effects on the target variable. For example, AcidIndex, Density,
and VolatileAcidity all have negative coefficients. While I'm not a wine
connoisseur myself, a highly dense wine seems unappealing at least.

Again, we note the huge reduction in deviance when going from the null
model to the full model--this speaks well to our model. Now, a further
question is if a Poisson model is appropriate here. A key condition for
Poisson is that the mean and variance of the response variable are
equal. We check this now:

\begin{verbatim}
## [1] 0.8837684
\end{verbatim}

So there certainly isn't over-dispersion. The under-dispersion is
somewhat surprising, but the value is close enough to 1, and certainly
close enough for a baseline model. We turn now to construct a new model

\subsubsection{Poisson Model 2
(Zero-Inflated)}\label{poisson-model-2-zero-inflated}

We observed in our last model that unrated wines perform especially
badly. Recall, though, we actually turned those values to unrated; they
were missing at first. What if, then, these values should actually be a
``0'' rating? If so, we might be able to use a model that both improves
accuracy and interpretability. The first step, then, is to create a new
column changing the unrated values to zeros.

\begin{table}[H]
\centering\centering
\caption{\label{tab:unnamed-chunk-21}STARS Value Counts}
\centering
\begin{tabular}[t]{l|r}
\hline
Var1 & Freq\\
\hline
0 & 11585\\
\hline
1 & 10755\\
\hline
2 & 12625\\
\hline
3 & 7635\\
\hline
4 & 2195\\
\hline
\end{tabular}
\end{table}

Immediately we see that this change leads to a large number of zeroes.
This is a strong indicator for considering a zero-inflated model,
especially given that a different process may well have led to a zero
rating than the process that led to the other ratings.

We start with creating a zero-inflated model using the same variables as
the most recent Poisson model as that provides a strong baseline:

\begin{verbatim}
## 
## Call:
## zeroinfl(formula = TARGET ~ Alcohol + I(Alcohol^2) + LabelAppeal + original_stars + 
##     AcidIndex + Chlorides + Density + FreeSulfurDioxide + Sulphates + 
##     TotalSulfurDioxide + VolatileAcidity + pH | original_stars, data = train_data_imputed, 
##     dist = "poisson")
## 
## Pearson residuals:
##      Min       1Q   Median       3Q      Max 
## -2.18614 -0.51971  0.01761  0.40950  2.87565 
## 
## Count model coefficients (poisson with log link):
##                      Estimate Std. Error z value Pr(>|z|)    
## (Intercept)         9.261e-01  1.102e-01   8.405  < 2e-16 ***
## Alcohol             6.327e-03  2.298e-03   2.754  0.00589 ** 
## I(Alcohol^2)        5.421e-06  1.008e-04   0.054  0.95714    
## LabelAppeal-1       3.886e-01  2.144e-02  18.123  < 2e-16 ***
## LabelAppeal0        6.625e-01  2.100e-02  31.554  < 2e-16 ***
## LabelAppeal1        8.558e-01  2.139e-02  40.000  < 2e-16 ***
## LabelAppeal2        1.019e+00  2.397e-02  42.529  < 2e-16 ***
## original_stars      9.769e-02  2.783e-03  35.106  < 2e-16 ***
## AcidIndex          -2.690e-02  2.669e-03 -10.078  < 2e-16 ***
## Chlorides          -2.682e-02  8.807e-03  -3.045  0.00232 ** 
## Density            -3.123e-01  1.062e-01  -2.941  0.00327 ** 
## FreeSulfurDioxide   3.227e-05  1.862e-05   1.733  0.08302 .  
## Sulphates           1.543e-04  3.024e-03   0.051  0.95930    
## TotalSulfurDioxide  1.201e-05  1.173e-05   1.024  0.30599    
## VolatileAcidity    -1.917e-02  3.549e-03  -5.402 6.58e-08 ***
## pH                  2.642e-03  4.097e-03   0.645  0.51889    
## 
## Zero-inflation model coefficients (binomial with logit link):
##                Estimate Std. Error z value Pr(>|z|)    
## (Intercept)     0.38027    0.01948   19.52   <2e-16 ***
## original_stars -2.18435    0.02824  -77.36   <2e-16 ***
## ---
## Signif. codes:  0 '***' 0.001 '**' 0.01 '*' 0.05 '.' 0.1 ' ' 1 
## 
## Number of iterations in BFGS optimization: 23 
## Log-likelihood: -7.297e+04 on 18 Df
\end{verbatim}

It's quite interesting how this one change changed the model fairly
significantly. We will finish removing variables, again using a backward
elimination process, and then add more commentary.

\begin{verbatim}
## 
## Call:
## zeroinfl(formula = TARGET ~ Alcohol + LabelAppeal + original_stars + 
##     AcidIndex + Chlorides + Density + VolatileAcidity | original_stars, 
##     data = train_data_imputed, dist = "poisson")
## 
## Pearson residuals:
##      Min       1Q   Median       3Q      Max 
## -2.18913 -0.51720  0.01805  0.40872  2.87298 
## 
## Count model coefficients (poisson with log link):
##                   Estimate Std. Error z value Pr(>|z|)    
## (Intercept)      0.9344984  0.1087370   8.594  < 2e-16 ***
## Alcohol          0.0064163  0.0007475   8.583  < 2e-16 ***
## LabelAppeal-1    0.3881827  0.0214393  18.106  < 2e-16 ***
## LabelAppeal0     0.6624722  0.0209931  31.557  < 2e-16 ***
## LabelAppeal1     0.8560520  0.0213906  40.020  < 2e-16 ***
## LabelAppeal2     1.0192949  0.0239632  42.536  < 2e-16 ***
## original_stars   0.0974396  0.0027784  35.071  < 2e-16 ***
## AcidIndex       -0.0271256  0.0026596 -10.199  < 2e-16 ***
## Chlorides       -0.0273112  0.0087975  -3.104  0.00191 ** 
## Density         -0.3075116  0.1061497  -2.897  0.00377 ** 
## VolatileAcidity -0.0191426  0.0035470  -5.397 6.78e-08 ***
## 
## Zero-inflation model coefficients (binomial with logit link):
##                Estimate Std. Error z value Pr(>|z|)    
## (Intercept)     0.38060    0.01947   19.55   <2e-16 ***
## original_stars -2.18398    0.02821  -77.41   <2e-16 ***
## ---
## Signif. codes:  0 '***' 0.001 '**' 0.01 '*' 0.05 '.' 0.1 ' ' 1 
## 
## Number of iterations in BFGS optimization: 19 
## Log-likelihood: -7.297e+04 on 13 Df
\end{verbatim}

There are many observations to be made. The first is that we used the
p-value to eliminate predictors that were not significant, and it is
striking that we were able to eliminate five variables once we switched
to a zero-inflated model. Second, the quadratic alcohol term was one of
those terms that was no longer significant. We were also able to
eliminate Sulphates, pH, and the SulfurDioxide variables. As for the
variables that persisted, the effects are not all that different: Label
Appeal and Stars have a positive effect, chemical properties have
negative effects. The key difference is that Alcohol now has a positive
effect, but that's intuitive now that the previously positively
impacting quadratic term is now removed.

As for the Stars variable (here called original\_stars), again higher
star ratings are associated with a higher log count of cases ordered.
It's also the case that wines with no star ratings are more likely to
have zero cases ordered.

Let's now generate predictions for the two Poisson models:

\begin{verbatim}
## MAE Poisson Model:  1.016386
\end{verbatim}

\begin{verbatim}
## RMSE Poisson Model:  1.275739
\end{verbatim}

\begin{verbatim}
## MAE ZIP Model:  0.9956513
\end{verbatim}

\begin{verbatim}
## RMSE ZIP Model:  1.284232
\end{verbatim}

Again, we will compare all models once all models are built, although
it's worth noting that both MAE and RMSE values are pretty close to 1,
which might be acceptable. But before we get ahead of ourselves, let's
build the next two models.

\subsection{Negative Binomial}\label{negative-binomial}

There is reason to believe that switching to a negative binomial model
will yield better results. Specifically, the negative binomial is
appropriate when we are working with count data that has over-dispersion
(the variance is greater than the mean). Now, it is true that earlier we
saw under-dispersion relative to what one Poisson model expects.
However, the truth is that it is really worthwhile to more get a direct
measure of the dispersion in the outcome variable, before even
modelling:

\begin{Shaded}
\begin{Highlighting}[]
\NormalTok{observed\_variance }\OtherTok{\textless{}{-}} \FunctionTok{var}\NormalTok{(train\_data\_imputed}\SpecialCharTok{$}\NormalTok{TARGET)}
\NormalTok{expected\_mean }\OtherTok{\textless{}{-}} \FunctionTok{mean}\NormalTok{(train\_data\_imputed}\SpecialCharTok{$}\NormalTok{TARGET)}
\FunctionTok{print}\NormalTok{(observed\_variance }\SpecialCharTok{/}\NormalTok{ expected\_mean)}
\end{Highlighting}
\end{Shaded}

\begin{verbatim}
## [1] 1.225267
\end{verbatim}

We see that this dispersion statistic is greater than 1. This suggests
that we ought to try a negative binomial model.

\subsubsection{Negative Binomial Model
1}\label{negative-binomial-model-1}

Much like earlier, we will start with a relatively simple model, at
first using all the variables as well as the interaction terms attempted
earlier, and then engaging in variable selection.

As a reminder, those additional variables are:

\begin{enumerate}
\def\labelenumi{\arabic{enumi}.}
\tightlist
\item
  Alcohol:LabelAppeal
\item
  STARS:Alcohol
\item
  Alcohol\^{}2
\end{enumerate}

(We omit LabelAppeal:STARS for the reason discussed earlier)

While it is true that most or all of these additional variables were not
significant in the previous two models, it does not follow that they'll
be insignificant in the negative binomial models.

\begin{verbatim}
## 
## Call:
## glm.nb(formula = TARGET ~ Alcohol + I(Alcohol^2) + LabelAppeal + 
##     STARS + AcidIndex + Chlorides + CitricAcid + Density + FixedAcidity + 
##     FreeSulfurDioxide + ResidualSugar + Sulphates + TotalSulfurDioxide + 
##     VolatileAcidity + pH + Alcohol:LabelAppeal + STARS:Alcohol, 
##     data = train_data_imputed, init.theta = 41159.01149, link = log)
## 
## Coefficients:
##                         Estimate Std. Error z value Pr(>|z|)    
## (Intercept)            1.478e+00  1.211e-01  12.204  < 2e-16 ***
## Alcohol                1.814e-03  5.755e-03   0.315 0.752556    
## I(Alcohol^2)           2.115e-04  1.027e-04   2.059 0.039449 *  
## LabelAppeal-1          3.036e-01  6.324e-02   4.801 1.58e-06 ***
## LabelAppeal0           5.301e-01  6.170e-02   8.592  < 2e-16 ***
## LabelAppeal1           6.403e-01  6.282e-02  10.192  < 2e-16 ***
## LabelAppeal2           9.446e-01  7.225e-02  13.074  < 2e-16 ***
## STARS2                 2.554e-01  2.234e-02  11.431  < 2e-16 ***
## STARS3                 4.111e-01  2.498e-02  16.455  < 2e-16 ***
## STARS4                 5.284e-01  3.691e-02  14.318  < 2e-16 ***
## STARSUnrated          -7.541e-01  3.044e-02 -24.773  < 2e-16 ***
## AcidIndex             -7.804e-02  2.445e-03 -31.912  < 2e-16 ***
## Chlorides             -3.866e-02  8.609e-03  -4.491 7.09e-06 ***
## CitricAcid             6.276e-03  3.184e-03   1.971 0.048712 *  
## Density               -3.394e-01  1.028e-01  -3.302 0.000959 ***
## FixedAcidity           1.370e-04  4.367e-04   0.314 0.753627    
## FreeSulfurDioxide      9.861e-05  1.840e-05   5.359 8.38e-08 ***
## ResidualSugar          1.619e-04  8.112e-05   1.996 0.045968 *  
## Sulphates             -8.979e-03  2.951e-03  -3.043 0.002344 ** 
## TotalSulfurDioxide     8.624e-05  1.179e-05   7.318 2.52e-13 ***
## VolatileAcidity       -3.567e-02  3.462e-03 -10.304  < 2e-16 ***
## pH                    -1.513e-02  3.988e-03  -3.793 0.000149 ***
## Alcohol:LabelAppeal-1 -3.225e-03  5.470e-03  -0.590 0.555504    
## Alcohol:LabelAppeal0  -6.556e-03  5.327e-03  -1.231 0.218412    
## Alcohol:LabelAppeal1  -3.875e-03  5.435e-03  -0.713 0.475857    
## Alcohol:LabelAppeal2  -1.860e-02  6.303e-03  -2.951 0.003163 ** 
## Alcohol:STARS2         6.739e-03  2.015e-03   3.345 0.000824 ***
## Alcohol:STARS3         3.168e-03  2.222e-03   1.426 0.153880    
## Alcohol:STARS4         2.983e-03  3.217e-03   0.927 0.353757    
## Alcohol:STARSUnrated   9.260e-04  2.781e-03   0.333 0.739116    
## ---
## Signif. codes:  0 '***' 0.001 '**' 0.01 '*' 0.05 '.' 0.1 ' ' 1
## 
## (Dispersion parameter for Negative Binomial(41159.01) family taken to be 1)
## 
##     Null deviance: 80039  on 44794  degrees of freedom
## Residual deviance: 48065  on 44765  degrees of freedom
## AIC: 159965
## 
## Number of Fisher Scoring iterations: 1
## 
## 
##               Theta:  41159 
##           Std. Err.:  18751 
## Warning while fitting theta: iteration limit reached 
## 
##  2 x log-likelihood:  -159903.1
\end{verbatim}

If the previous model was any indication, there's likely high
collinearity. Let's check:

\begin{table}[H]
\centering\centering
\caption{\label{tab:unnamed-chunk-26}VIF Values for NB Model}
\centering
\begin{tabu} to \linewidth {>{\raggedright}X>{\raggedleft}X>{\raggedleft}X>{\raggedleft}X>{\raggedleft}X>{\raggedleft}X>{\raggedleft}X>{\raggedleft}X}
\hline
Term & VIF & VIF\_CI\_low & VIF\_CI\_high & SE\_factor & Tolerance & Tolerance\_CI\_low & Tolerance\_CI\_high\\
\hline
Alcohol & 62.614401 & 61.475476 & 63.774775 & 7.912926 & 0.0159708 & 0.0156802 & 0.0162666\\
\hline
I(Alcohol\textasciicircum{}2) & 9.975600 & 9.801932 & 10.152694 & 3.158417 & 0.1002446 & 0.0984960 & 0.1020207\\
\hline
LabelAppeal & 9648.864165 & 9471.939272 & 9829.094160 & 98.228632 & 0.0001036 & 0.0001017 & 0.0001056\\
\hline
STARS & 8808.209861 & 8646.700301 & 8972.736567 & 93.852064 & 0.0001135 & 0.0001114 & 0.0001157\\
\hline
AcidIndex & 1.063458 & 1.053841 & 1.074792 & 1.031241 & 0.9403290 & 0.9304130 & 0.9489096\\
\hline
Chlorides & 1.005769 & 1.001139 & 1.029227 & 1.002880 & 0.9942643 & 0.9716033 & 0.9988627\\
\hline
CitricAcid & 1.008438 & 1.002766 & 1.025737 & 1.004210 & 0.9916325 & 0.9749088 & 0.9972411\\
\hline
Density & 1.005458 & 1.000983 & 1.030296 & 1.002725 & 0.9945714 & 0.9705952 & 0.9990176\\
\hline
FixedAcidity & 1.024270 & 1.016272 & 1.036199 & 1.012062 & 0.9763050 & 0.9650655 & 0.9839884\\
\hline
FreeSulfurDioxide & 1.006183 & 1.001359 & 1.028133 & 1.003087 & 0.9938549 & 0.9726368 & 0.9986429\\
\hline
ResidualSugar & 1.003295 & 1.000195 & 1.055652 & 1.001646 & 0.9967154 & 0.9472821 & 0.9998049\\
\hline
Sulphates & 1.004457 & 1.000549 & 1.036204 & 1.002226 & 0.9955626 & 0.9650611 & 0.9994516\\
\hline
TotalSulfurDioxide & 1.005488 & 1.000998 & 1.030182 & 1.002740 & 0.9945418 & 0.9707025 & 0.9990031\\
\hline
VolatileAcidity & 1.007461 & 1.002119 & 1.026271 & 1.003724 & 0.9925939 & 0.9744014 & 0.9978853\\
\hline
pH & 1.008664 & 1.002923 & 1.025681 & 1.004323 & 0.9914107 & 0.9749619 & 0.9970857\\
\hline
Alcohol:LabelAppeal & 61459.298562 & 60332.310033 & 62607.339232 & 247.909860 & 0.0000163 & 0.0000160 & 0.0000166\\
\hline
Alcohol:STARS & 13147.088803 & 12906.015905 & 13392.665078 & 114.660755 & 0.0000761 & 0.0000747 & 0.0000775\\
\hline
\end{tabu}
\end{table}

Yet again, there are some extraordinarily high VIF values, likely
because of the interaction terms. Again, then, we consult the VIF,
remodel, and repeat until there are no variables with high collinearity.

\begin{verbatim}
## 
## Call:
## glm.nb(formula = TARGET ~ Alcohol + I(Alcohol^2) + LabelAppeal + 
##     STARS + AcidIndex + Chlorides + CitricAcid + Density + FixedAcidity + 
##     FreeSulfurDioxide + ResidualSugar + Sulphates + TotalSulfurDioxide + 
##     VolatileAcidity + pH, data = train_data_imputed, init.theta = 41131.91211, 
##     link = log)
## 
## Coefficients:
##                      Estimate Std. Error z value Pr(>|z|)    
## (Intercept)         1.505e+00  1.065e-01  14.134  < 2e-16 ***
## Alcohol            -4.228e-04  2.289e-03  -0.185 0.853454    
## I(Alcohol^2)        2.183e-04  1.023e-04   2.134 0.032803 *  
## LabelAppeal-1       2.688e-01  2.074e-02  12.961  < 2e-16 ***
## LabelAppeal0        4.599e-01  2.027e-02  22.689  < 2e-16 ***
## LabelAppeal1        5.983e-01  2.060e-02  29.040  < 2e-16 ***
## LabelAppeal2        7.446e-01  2.318e-02  32.126  < 2e-16 ***
## STARS2              3.258e-01  7.656e-03  42.552  < 2e-16 ***
## STARS3              4.437e-01  8.361e-03  53.073  < 2e-16 ***
## STARS4              5.586e-01  1.150e-02  48.562  < 2e-16 ***
## STARSUnrated       -7.445e-01  1.046e-02 -71.178  < 2e-16 ***
## AcidIndex          -7.794e-02  2.442e-03 -31.914  < 2e-16 ***
## Chlorides          -3.930e-02  8.604e-03  -4.568 4.92e-06 ***
## CitricAcid          6.049e-03  3.183e-03   1.900 0.057412 .  
## Density            -3.418e-01  1.027e-01  -3.328 0.000875 ***
## FixedAcidity        1.468e-04  4.366e-04   0.336 0.736665    
## FreeSulfurDioxide   9.914e-05  1.840e-05   5.387 7.16e-08 ***
## ResidualSugar       1.618e-04  8.110e-05   1.995 0.046016 *  
## Sulphates          -8.780e-03  2.950e-03  -2.976 0.002916 ** 
## TotalSulfurDioxide  8.684e-05  1.178e-05   7.374 1.66e-13 ***
## VolatileAcidity    -3.535e-02  3.460e-03 -10.218  < 2e-16 ***
## pH                 -1.541e-02  3.987e-03  -3.864 0.000111 ***
## ---
## Signif. codes:  0 '***' 0.001 '**' 0.01 '*' 0.05 '.' 0.1 ' ' 1
## 
## (Dispersion parameter for Negative Binomial(41131.91) family taken to be 1)
## 
##     Null deviance: 80039  on 44794  degrees of freedom
## Residual deviance: 48097  on 44773  degrees of freedom
## AIC: 159982
## 
## Number of Fisher Scoring iterations: 1
## 
## 
##               Theta:  41132 
##           Std. Err.:  18745 
## Warning while fitting theta: iteration limit reached 
## 
##  2 x log-likelihood:  -159935.6
\end{verbatim}

There are still a few columns that don't have significant predictors,
and so again we consult the p-values to remove them one at a time:

\begin{verbatim}
## 
## Call:
## glm.nb(formula = TARGET ~ Alcohol + I(Alcohol^2) + LabelAppeal + 
##     STARS + AcidIndex + Chlorides + CitricAcid + Density + FreeSulfurDioxide + 
##     ResidualSugar + Sulphates + TotalSulfurDioxide + VolatileAcidity + 
##     pH, data = train_data_imputed, init.theta = 41133.96599, 
##     link = log)
## 
## Coefficients:
##                      Estimate Std. Error z value Pr(>|z|)    
## (Intercept)         1.504e+00  1.065e-01  14.131  < 2e-16 ***
## Alcohol            -4.237e-04  2.289e-03  -0.185 0.853166    
## I(Alcohol^2)        2.184e-04  1.023e-04   2.135 0.032731 *  
## LabelAppeal-1       2.687e-01  2.074e-02  12.959  < 2e-16 ***
## LabelAppeal0        4.599e-01  2.027e-02  22.687  < 2e-16 ***
## LabelAppeal1        5.982e-01  2.060e-02  29.038  < 2e-16 ***
## LabelAppeal2        7.445e-01  2.318e-02  32.124  < 2e-16 ***
## STARS2              3.258e-01  7.656e-03  42.551  < 2e-16 ***
## STARS3              4.438e-01  8.360e-03  53.080  < 2e-16 ***
## STARS4              5.586e-01  1.150e-02  48.562  < 2e-16 ***
## STARSUnrated       -7.445e-01  1.046e-02 -71.178  < 2e-16 ***
## AcidIndex          -7.783e-02  2.418e-03 -32.182  < 2e-16 ***
## Chlorides          -3.934e-02  8.603e-03  -4.573 4.81e-06 ***
## CitricAcid          6.072e-03  3.183e-03   1.908 0.056420 .  
## Density            -3.413e-01  1.027e-01  -3.323 0.000890 ***
## FreeSulfurDioxide   9.916e-05  1.840e-05   5.388 7.11e-08 ***
## ResidualSugar       1.616e-04  8.110e-05   1.993 0.046291 *  
## Sulphates          -8.753e-03  2.949e-03  -2.969 0.002992 ** 
## TotalSulfurDioxide  8.678e-05  1.178e-05   7.369 1.72e-13 ***
## VolatileAcidity    -3.534e-02  3.460e-03 -10.215  < 2e-16 ***
## pH                 -1.539e-02  3.987e-03  -3.861 0.000113 ***
## ---
## Signif. codes:  0 '***' 0.001 '**' 0.01 '*' 0.05 '.' 0.1 ' ' 1
## 
## (Dispersion parameter for Negative Binomial(41133.97) family taken to be 1)
## 
##     Null deviance: 80039  on 44794  degrees of freedom
## Residual deviance: 48097  on 44774  degrees of freedom
## AIC: 159980
## 
## Number of Fisher Scoring iterations: 1
## 
## 
##               Theta:  41134 
##           Std. Err.:  18747 
## Warning while fitting theta: iteration limit reached 
## 
##  2 x log-likelihood:  -159935.7
\end{verbatim}

\begin{verbatim}
## 
## Call:
## glm.nb(formula = TARGET ~ Alcohol + I(Alcohol^2) + LabelAppeal + 
##     STARS + AcidIndex + Chlorides + Density + FreeSulfurDioxide + 
##     ResidualSugar + Sulphates + TotalSulfurDioxide + VolatileAcidity + 
##     pH, data = train_data_imputed, init.theta = 41133.47465, 
##     link = log)
## 
## Coefficients:
##                      Estimate Std. Error z value Pr(>|z|)    
## (Intercept)         1.507e+00  1.064e-01  14.154  < 2e-16 ***
## Alcohol            -4.679e-04  2.289e-03  -0.204 0.838026    
## I(Alcohol^2)        2.214e-04  1.023e-04   2.165 0.030391 *  
## LabelAppeal-1       2.683e-01  2.073e-02  12.938  < 2e-16 ***
## LabelAppeal0        4.595e-01  2.027e-02  22.667  < 2e-16 ***
## LabelAppeal1        5.979e-01  2.060e-02  29.022  < 2e-16 ***
## LabelAppeal2        7.445e-01  2.318e-02  32.125  < 2e-16 ***
## STARS2              3.260e-01  7.655e-03  42.579  < 2e-16 ***
## STARS3              4.438e-01  8.360e-03  53.085  < 2e-16 ***
## STARS4              5.589e-01  1.150e-02  48.592  < 2e-16 ***
## STARSUnrated       -7.447e-01  1.046e-02 -71.196  < 2e-16 ***
## AcidIndex          -7.752e-02  2.413e-03 -32.128  < 2e-16 ***
## Chlorides          -3.967e-02  8.601e-03  -4.612 4.00e-06 ***
## Density            -3.436e-01  1.027e-01  -3.345 0.000821 ***
## FreeSulfurDioxide   9.935e-05  1.840e-05   5.398 6.72e-08 ***
## ResidualSugar       1.585e-04  8.107e-05   1.955 0.050578 .  
## Sulphates          -8.817e-03  2.949e-03  -2.990 0.002789 ** 
## TotalSulfurDioxide  8.687e-05  1.177e-05   7.378 1.61e-13 ***
## VolatileAcidity    -3.541e-02  3.459e-03 -10.235  < 2e-16 ***
## pH                 -1.541e-02  3.987e-03  -3.865 0.000111 ***
## ---
## Signif. codes:  0 '***' 0.001 '**' 0.01 '*' 0.05 '.' 0.1 ' ' 1
## 
## (Dispersion parameter for Negative Binomial(41133.47) family taken to be 1)
## 
##     Null deviance: 80039  on 44794  degrees of freedom
## Residual deviance: 48101  on 44775  degrees of freedom
## AIC: 159981
## 
## Number of Fisher Scoring iterations: 1
## 
## 
##               Theta:  41133 
##           Std. Err.:  18748 
## Warning while fitting theta: iteration limit reached 
## 
##  2 x log-likelihood:  -159939.4
\end{verbatim}

\begin{verbatim}
## 
## Call:
## glm.nb(formula = TARGET ~ Alcohol + I(Alcohol^2) + LabelAppeal + 
##     STARS + AcidIndex + Chlorides + Density + FreeSulfurDioxide + 
##     Sulphates + TotalSulfurDioxide + VolatileAcidity + pH, data = train_data_imputed, 
##     init.theta = 41129.92506, link = log)
## 
## Coefficients:
##                      Estimate Std. Error z value Pr(>|z|)    
## (Intercept)         1.507e+00  1.065e-01  14.158  < 2e-16 ***
## Alcohol            -4.549e-04  2.289e-03  -0.199 0.842451    
## I(Alcohol^2)        2.199e-04  1.022e-04   2.151 0.031464 *  
## LabelAppeal-1       2.685e-01  2.073e-02  12.948  < 2e-16 ***
## LabelAppeal0        4.595e-01  2.027e-02  22.668  < 2e-16 ***
## LabelAppeal1        5.980e-01  2.060e-02  29.028  < 2e-16 ***
## LabelAppeal2        7.450e-01  2.317e-02  32.149  < 2e-16 ***
## STARS2              3.262e-01  7.654e-03  42.617  < 2e-16 ***
## STARS3              4.440e-01  8.360e-03  53.111  < 2e-16 ***
## STARS4              5.589e-01  1.150e-02  48.594  < 2e-16 ***
## STARSUnrated       -7.447e-01  1.046e-02 -71.200  < 2e-16 ***
## AcidIndex          -7.751e-02  2.413e-03 -32.128  < 2e-16 ***
## Chlorides          -3.975e-02  8.601e-03  -4.622 3.80e-06 ***
## Density            -3.440e-01  1.027e-01  -3.349 0.000811 ***
## FreeSulfurDioxide   9.996e-05  1.840e-05   5.432 5.56e-08 ***
## Sulphates          -8.901e-03  2.949e-03  -3.019 0.002539 ** 
## TotalSulfurDioxide  8.728e-05  1.177e-05   7.413 1.23e-13 ***
## VolatileAcidity    -3.551e-02  3.459e-03 -10.266  < 2e-16 ***
## pH                 -1.527e-02  3.986e-03  -3.830 0.000128 ***
## ---
## Signif. codes:  0 '***' 0.001 '**' 0.01 '*' 0.05 '.' 0.1 ' ' 1
## 
## (Dispersion parameter for Negative Binomial(41129.93) family taken to be 1)
## 
##     Null deviance: 80039  on 44794  degrees of freedom
## Residual deviance: 48105  on 44776  degrees of freedom
## AIC: 159983
## 
## Number of Fisher Scoring iterations: 1
## 
## 
##               Theta:  41130 
##           Std. Err.:  18745 
## Warning while fitting theta: iteration limit reached 
## 
##  2 x log-likelihood:  -159943.2
\end{verbatim}

And so we arrive at a negative binomial model with some very interesting
results. In particular, the estimates and standard errors are nearly
identical as with the simple Poisson model! In fact, even the AIC values
are quite similar.

\begin{table}[H]
\centering\centering
\caption{\label{tab:comparison table}Comparison of Simple Poisson and Negatiive Binomial Models}
\centering
\begin{tabular}[t]{l|r|r|r|r}
\hline
Term & Poisson\_Estimate & NB\_Estimate & Poisson\_Std\_Error & NB\_Std\_Error\\
\hline
(Intercept) & 1.5071 & 1.5071 & 0.1064 & 0.1065\\
\hline
Alcohol & -0.0005 & -0.0005 & 0.0023 & 0.0023\\
\hline
I(Alcohol\textasciicircum{}2) & 0.0002 & 0.0002 & 0.0001 & 0.0001\\
\hline
LabelAppeal-1 & 0.2685 & 0.2685 & 0.0207 & 0.0207\\
\hline
LabelAppeal0 & 0.4595 & 0.4595 & 0.0203 & 0.0203\\
\hline
LabelAppeal1 & 0.5980 & 0.5980 & 0.0206 & 0.0206\\
\hline
LabelAppeal2 & 0.7450 & 0.7450 & 0.0232 & 0.0232\\
\hline
STARS2 & 0.3262 & 0.3262 & 0.0077 & 0.0077\\
\hline
STARS3 & 0.4440 & 0.4440 & 0.0084 & 0.0084\\
\hline
STARS4 & 0.5589 & 0.5589 & 0.0115 & 0.0115\\
\hline
STARSUnrated & -0.7447 & -0.7447 & 0.0105 & 0.0105\\
\hline
AcidIndex & -0.0775 & -0.0775 & 0.0024 & 0.0024\\
\hline
Chlorides & -0.0398 & -0.0398 & 0.0086 & 0.0086\\
\hline
Density & -0.3439 & -0.3440 & 0.1027 & 0.1027\\
\hline
FreeSulfurDioxide & 0.0001 & 0.0001 & 0.0000 & 0.0000\\
\hline
Sulphates & -0.0089 & -0.0089 & 0.0029 & 0.0029\\
\hline
TotalSulfurDioxide & 0.0001 & 0.0001 & 0.0000 & 0.0000\\
\hline
VolatileAcidity & -0.0355 & -0.0355 & 0.0035 & 0.0035\\
\hline
pH & -0.0153 & -0.0153 & 0.0040 & 0.0040\\
\hline
\end{tabular}
\end{table}

This is striking at first, and it is \emph{possible} that it's due to
convergence issues with the negative binomial model. It also seems that
the dispersion is quite close to Poisson Assumptions (i.e.~mean
approximately equal to the variance), which is supported by the very
high value of theta. Again, model selection will occur after all the
models are built, but it certainly seems that there is no reason to
accept this model over the simple Poisson one.

Before we give up entirely on a negative binomial model, though, let's
try a zero-inflated model as we did earlier:

\subsubsection{Negative Binomial Model 2
(Zero-Inflated)}\label{negative-binomial-model-2-zero-inflated}

We mimic the approach from earlier, starting by creating a zero-inflated
model using the same variables as the most recent Negative Binomial
model as that provides a strong baseline:

\begin{verbatim}
## 
## Call:
## zeroinfl(formula = TARGET ~ Alcohol + I(Alcohol^2) + LabelAppeal + original_stars + 
##     AcidIndex + Chlorides + Density + FreeSulfurDioxide + Sulphates + 
##     TotalSulfurDioxide + VolatileAcidity + pH | original_stars, data = train_data_imputed, 
##     dist = "negbin")
## 
## Pearson residuals:
##      Min       1Q   Median       3Q      Max 
## -2.18618 -0.51967  0.01759  0.40947  2.87559 
## 
## Count model coefficients (negbin with log link):
##                      Estimate Std. Error z value Pr(>|z|)    
## (Intercept)         9.255e-01  1.102e-01   8.400  < 2e-16 ***
## Alcohol             6.329e-03  2.298e-03   2.755  0.00588 ** 
## I(Alcohol^2)        5.380e-06  1.008e-04   0.053  0.95745    
## LabelAppeal-1       3.887e-01  2.144e-02  18.126  < 2e-16 ***
## LabelAppeal0        6.626e-01  2.100e-02  31.558  < 2e-16 ***
## LabelAppeal1        8.559e-01  2.139e-02  40.004  < 2e-16 ***
## LabelAppeal2        1.019e+00  2.397e-02  42.532  < 2e-16 ***
## original_stars      9.768e-02  2.783e-03  35.103  < 2e-16 ***
## AcidIndex          -2.690e-02  2.669e-03 -10.080  < 2e-16 ***
## Chlorides          -2.682e-02  8.807e-03  -3.045  0.00233 ** 
## Density            -3.118e-01  1.062e-01  -2.937  0.00332 ** 
## FreeSulfurDioxide   3.227e-05  1.862e-05   1.733  0.08304 .  
## Sulphates           1.622e-04  3.024e-03   0.054  0.95722    
## TotalSulfurDioxide  1.201e-05  1.173e-05   1.024  0.30570    
## VolatileAcidity    -1.917e-02  3.549e-03  -5.402 6.58e-08 ***
## pH                  2.646e-03  4.097e-03   0.646  0.51831    
## Log(theta)          1.792e+01  1.489e+00  12.039  < 2e-16 ***
## 
## Zero-inflation model coefficients (binomial with logit link):
##                Estimate Std. Error z value Pr(>|z|)    
## (Intercept)     0.38026    0.01948   19.52   <2e-16 ***
## original_stars -2.18434    0.02824  -77.36   <2e-16 ***
## ---
## Signif. codes:  0 '***' 0.001 '**' 0.01 '*' 0.05 '.' 0.1 ' ' 1 
## 
## Theta = 60886108.1239 
## Number of iterations in BFGS optimization: 66 
## Log-likelihood: -7.297e+04 on 19 Df
\end{verbatim}

Notice again, the extremely high theta value suggests that the variance
is very close to that of a Poisson distribution. The similarites in
estimates and errors are thus predictable. Still, we'll complete the
backward elimination before doing a proper comparison with the Poisson
zero-inflated model.

\begin{verbatim}
## 
## Call:
## zeroinfl(formula = TARGET ~ Alcohol + LabelAppeal + original_stars + 
##     AcidIndex + Chlorides + Density + VolatileAcidity | original_stars, 
##     data = train_data_imputed, dist = "negbin")
## 
## Pearson residuals:
##     Min      1Q  Median      3Q     Max 
## -2.1894 -0.5172  0.0179  0.4088  2.8725 
## 
## Count model coefficients (negbin with log link):
##                   Estimate Std. Error z value Pr(>|z|)    
## (Intercept)      0.9341047  0.1087374   8.590  < 2e-16 ***
## Alcohol          0.0064161  0.0007475   8.583  < 2e-16 ***
## LabelAppeal-1    0.3882117  0.0214397  18.107  < 2e-16 ***
## LabelAppeal0     0.6625338  0.0209936  31.559  < 2e-16 ***
## LabelAppeal1     0.8562690  0.0213910  40.029  < 2e-16 ***
## LabelAppeal2     1.0194905  0.0239634  42.544  < 2e-16 ***
## original_stars   0.0974036  0.0027784  35.058  < 2e-16 ***
## AcidIndex       -0.0271432  0.0026596 -10.206  < 2e-16 ***
## Chlorides       -0.0273333  0.0087975  -3.107  0.00189 ** 
## Density         -0.3069899  0.1061500  -2.892  0.00383 ** 
## VolatileAcidity -0.0191950  0.0035470  -5.412 6.25e-08 ***
## Log(theta)      12.1996160  1.9412019   6.285 3.29e-10 ***
## 
## Zero-inflation model coefficients (binomial with logit link):
##                Estimate Std. Error z value Pr(>|z|)    
## (Intercept)     0.38079    0.01947   19.56   <2e-16 ***
## original_stars -2.18488    0.02823  -77.39   <2e-16 ***
## ---
## Signif. codes:  0 '***' 0.001 '**' 0.01 '*' 0.05 '.' 0.1 ' ' 1 
## 
## Theta = 198712.8242 
## Number of iterations in BFGS optimization: 30 
## Log-likelihood: -7.297e+04 on 14 Df
\end{verbatim}

Note, these are \emph{again} the same coefficients as in the
zero-inflated Poisson--this despite the fact that we only used the
p-values to guide our variable selection at this latter phase.

\begin{table}[H]
\centering\centering
\caption{\label{tab:comparison of zero inflated models}Comparison of Zero-Inflated Poisson and Negatiive Binomial Models}
\centering
\begin{tabular}[t]{l|r|r|r|r}
\hline
Term & ZIP\_Estimate & ZINB\_Estimate & ZIP\_Std\_Error & ZINB\_Std\_Error\\
\hline
(Intercept) & 0.9261 & 0.9255 & 0.1102 & 0.1102\\
\hline
Alcohol & 0.0063 & 0.0063 & 0.0023 & 0.0023\\
\hline
I(Alcohol\textasciicircum{}2) & 0.0000 & 0.0000 & 0.0001 & 0.0001\\
\hline
LabelAppeal-1 & 0.3886 & 0.3887 & 0.0214 & 0.0214\\
\hline
LabelAppeal0 & 0.6625 & 0.6626 & 0.0210 & 0.0210\\
\hline
LabelAppeal1 & 0.8558 & 0.8559 & 0.0214 & 0.0214\\
\hline
LabelAppeal2 & 1.0194 & 1.0195 & 0.0240 & 0.0240\\
\hline
original\_stars & 0.0977 & 0.0977 & 0.0028 & 0.0028\\
\hline
AcidIndex & -0.0269 & -0.0269 & 0.0027 & 0.0027\\
\hline
Chlorides & -0.0268 & -0.0268 & 0.0088 & 0.0088\\
\hline
Density & -0.3123 & -0.3118 & 0.1062 & 0.1062\\
\hline
FreeSulfurDioxide & 0.0000 & 0.0000 & 0.0000 & 0.0000\\
\hline
Sulphates & 0.0002 & 0.0002 & 0.0030 & 0.0030\\
\hline
TotalSulfurDioxide & 0.0000 & 0.0000 & 0.0000 & 0.0000\\
\hline
VolatileAcidity & -0.0192 & -0.0192 & 0.0035 & 0.0035\\
\hline
pH & 0.0026 & 0.0026 & 0.0041 & 0.0041\\
\hline
Log(theta) & NA & 17.9245 & NA & 1.4889\\
\hline
\end{tabular}
\end{table}

So again, we are looking at nearly identical statistics from the
zero-inflated Poisson to the zero-inflated negative binomial. Also
again, we should prefer the zero-inflated Poisson to the zero-inflated
negative binomial, since the distrubtion appars to b eclose neough to a
Poisson.

And what of the comparison beetwene the two ngative binomial models? Of
course, it is going to look extremeely similar to the comparison btween
the two Poisson models. Still, we add it below for sake of
completeneess:

\begin{verbatim}
## MAE Negative Binomial Model:  1.016387
\end{verbatim}

\begin{verbatim}
## RMSE Negative Binomial Model:  1.27574
\end{verbatim}

\begin{verbatim}
## MAE Zero-Inflated Negative Binomial Model:  0.9956453
\end{verbatim}

\begin{verbatim}
## RMSE Zero-Inflated Negative Binomial Model:  1.28426
\end{verbatim}

And indeed, the comparison is extremely similar as to earlier.

\subsubsection{Multiple Linear
Regression}\label{multiple-linear-regression}

We will look at a more direct approach with multiple linear regression
using our normalized, transformed variables.

Looking at each predictor variable, we see that each predictor besides
\texttt{FixedAcidity\_transformed} are statistically significant within
a 95\% confidence level We also see that our \(Adj. R^2 = 0.5405\), is
where our model accounts on average for 54\% of the variation of the
\texttt{TARGET} variable.

\begin{verbatim}
## 
## Call:
## lm(formula = as.numeric(as.character(TARGET)) ~ AcidIndex + FixedAcidity_transformed + 
##     VolatileAcidity_transformed + CitricAcid_transformed + ResidualSugar_transformed + 
##     Chlorides_transformed + FreeSulfurDioxide_transformed + TotalSulfurDioxide_transformed + 
##     Density_transformed + pH_transformed + Sulphates_transformed + 
##     Alcohol_transformed + STARS.1 + STARS.2 + STARS.3 + STARS.4 + 
##     STARS.Unrated + LabelAppeal, data = train_data_prepped)
## 
## Residuals:
##     Min      1Q  Median      3Q     Max 
## -4.7226 -0.8468  0.0138  0.8399  6.1074 
## 
## Coefficients: (1 not defined because of singularities)
##                                 Estimate Std. Error t value Pr(>|t|)    
## (Intercept)                     1.968728   0.051007  38.597  < 2e-16 ***
## AcidIndex                      -0.190274   0.004985 -38.169  < 2e-16 ***
## FixedAcidity_transformed        0.009494   0.006384   1.487 0.136974    
## VolatileAcidity_transformed    -0.105261   0.006209 -16.954  < 2e-16 ***
## CitricAcid_transformed          0.032272   0.006216   5.192 2.09e-07 ***
## ResidualSugar_transformed       0.020849   0.006188   3.369 0.000754 ***
## Chlorides_transformed          -0.062869   0.006205 -10.132  < 2e-16 ***
## FreeSulfurDioxide_transformed   0.062972   0.006206  10.147  < 2e-16 ***
## TotalSulfurDioxide_transformed  0.066956   0.006205  10.790  < 2e-16 ***
## Density_transformed            -0.051618   0.006209  -8.314  < 2e-16 ***
## pH_transformed                 -0.040865   0.006203  -6.588 4.52e-11 ***
## Sulphates_transformed          -0.028562   0.006191  -4.613 3.97e-06 ***
## Alcohol_transformed             0.061323   0.006214   9.868  < 2e-16 ***
## STARS.1                         1.311245   0.017628  74.384  < 2e-16 ***
## STARS.2                         2.354115   0.017173 137.080  < 2e-16 ***
## STARS.3                         2.906541   0.020015 145.219  < 2e-16 ***
## STARS.4                         3.564148   0.031432 113.393  < 2e-16 ***
## STARS.Unrated                         NA         NA      NA       NA    
## LabelAppeal-1                   0.413529   0.033736  12.258  < 2e-16 ***
## LabelAppeal0                    0.886686   0.032975  26.890  < 2e-16 ***
## LabelAppeal1                    1.375899   0.034432  39.960  < 2e-16 ***
## LabelAppeal2                    1.983104   0.045480  43.604  < 2e-16 ***
## ---
## Signif. codes:  0 '***' 0.001 '**' 0.01 '*' 0.05 '.' 0.1 ' ' 1
## 
## Residual standard error: 1.306 on 44774 degrees of freedom
## Multiple R-squared:  0.5407, Adjusted R-squared:  0.5405 
## F-statistic:  2635 on 20 and 44774 DF,  p-value: < 2.2e-16
\end{verbatim}

Calculating the RMSE for our training multiple regression model, we
obtain an \(RMSE = 1.3057\)

\begin{verbatim}
## [1] 1.305704
\end{verbatim}

\subsubsection{Stepwise Regression}\label{stepwise-regression}

We will create a stepwise regression model, to perform forward and
backward elimination on our multiple linear regression attempt.

The results show that we strictly removed the \texttt{STARS.Unrated}
column which makes sense, as it does not provide additional information
as the other STARS columns accounts for it. The \(Adj. R^2 = 0.5405\)
which stayed the same as before, and the RMSE stayed the same as well.

\begin{verbatim}
## Start:  AIC=23939.46
## as.numeric(as.character(TARGET)) ~ AcidIndex + FixedAcidity_transformed + 
##     VolatileAcidity_transformed + CitricAcid_transformed + ResidualSugar_transformed + 
##     Chlorides_transformed + FreeSulfurDioxide_transformed + TotalSulfurDioxide_transformed + 
##     Density_transformed + pH_transformed + Sulphates_transformed + 
##     Alcohol_transformed + STARS.1 + STARS.2 + STARS.3 + STARS.4 + 
##     STARS.Unrated + LabelAppeal
## 
## 
## Step:  AIC=23939.46
## as.numeric(as.character(TARGET)) ~ AcidIndex + FixedAcidity_transformed + 
##     VolatileAcidity_transformed + CitricAcid_transformed + ResidualSugar_transformed + 
##     Chlorides_transformed + FreeSulfurDioxide_transformed + TotalSulfurDioxide_transformed + 
##     Density_transformed + pH_transformed + Sulphates_transformed + 
##     Alcohol_transformed + STARS.1 + STARS.2 + STARS.3 + STARS.4 + 
##     LabelAppeal
## 
##                                  Df Sum of Sq    RSS   AIC
## <none>                                         76369 23939
## - FixedAcidity_transformed        1         4  76373 23940
## - ResidualSugar_transformed       1        19  76389 23949
## - Sulphates_transformed           1        36  76406 23959
## - CitricAcid_transformed          1        46  76415 23964
## - pH_transformed                  1        74  76443 23981
## - Density_transformed             1       118  76487 24007
## - Alcohol_transformed             1       166  76535 24035
## - Chlorides_transformed           1       175  76544 24040
## - FreeSulfurDioxide_transformed   1       176  76545 24040
## - TotalSulfurDioxide_transformed  1       199  76568 24054
## - VolatileAcidity_transformed     1       490  76860 24224
## - AcidIndex                       1      2485  78854 25372
## - LabelAppeal                     4      7629  83999 28197
## - STARS.1                         1      9437  85807 29157
## - STARS.4                         1     21931  98301 35246
## - STARS.2                         1     32051 108420 39635
## - STARS.3                         1     35970 112339 41226
\end{verbatim}

\begin{verbatim}
## 
## Call:
## lm(formula = as.numeric(as.character(TARGET)) ~ AcidIndex + FixedAcidity_transformed + 
##     VolatileAcidity_transformed + CitricAcid_transformed + ResidualSugar_transformed + 
##     Chlorides_transformed + FreeSulfurDioxide_transformed + TotalSulfurDioxide_transformed + 
##     Density_transformed + pH_transformed + Sulphates_transformed + 
##     Alcohol_transformed + STARS.1 + STARS.2 + STARS.3 + STARS.4 + 
##     LabelAppeal, data = train_data_prepped)
## 
## Residuals:
##     Min      1Q  Median      3Q     Max 
## -4.7226 -0.8468  0.0138  0.8399  6.1074 
## 
## Coefficients:
##                                 Estimate Std. Error t value Pr(>|t|)    
## (Intercept)                     1.968728   0.051007  38.597  < 2e-16 ***
## AcidIndex                      -0.190274   0.004985 -38.169  < 2e-16 ***
## FixedAcidity_transformed        0.009494   0.006384   1.487 0.136974    
## VolatileAcidity_transformed    -0.105261   0.006209 -16.954  < 2e-16 ***
## CitricAcid_transformed          0.032272   0.006216   5.192 2.09e-07 ***
## ResidualSugar_transformed       0.020849   0.006188   3.369 0.000754 ***
## Chlorides_transformed          -0.062869   0.006205 -10.132  < 2e-16 ***
## FreeSulfurDioxide_transformed   0.062972   0.006206  10.147  < 2e-16 ***
## TotalSulfurDioxide_transformed  0.066956   0.006205  10.790  < 2e-16 ***
## Density_transformed            -0.051618   0.006209  -8.314  < 2e-16 ***
## pH_transformed                 -0.040865   0.006203  -6.588 4.52e-11 ***
## Sulphates_transformed          -0.028562   0.006191  -4.613 3.97e-06 ***
## Alcohol_transformed             0.061323   0.006214   9.868  < 2e-16 ***
## STARS.1                         1.311245   0.017628  74.384  < 2e-16 ***
## STARS.2                         2.354115   0.017173 137.080  < 2e-16 ***
## STARS.3                         2.906541   0.020015 145.219  < 2e-16 ***
## STARS.4                         3.564148   0.031432 113.393  < 2e-16 ***
## LabelAppeal-1                   0.413529   0.033736  12.258  < 2e-16 ***
## LabelAppeal0                    0.886686   0.032975  26.890  < 2e-16 ***
## LabelAppeal1                    1.375899   0.034432  39.960  < 2e-16 ***
## LabelAppeal2                    1.983104   0.045480  43.604  < 2e-16 ***
## ---
## Signif. codes:  0 '***' 0.001 '**' 0.01 '*' 0.05 '.' 0.1 ' ' 1
## 
## Residual standard error: 1.306 on 44774 degrees of freedom
## Multiple R-squared:  0.5407, Adjusted R-squared:  0.5405 
## F-statistic:  2635 on 20 and 44774 DF,  p-value: < 2.2e-16
\end{verbatim}

\begin{verbatim}
## [1] 1.305704
\end{verbatim}

\subsubsection{Model Selection}\label{model-selection}

Let's now look at all the models compared to each other based on their
results with using the test data. The best model in comparison to the
others is the original theory of using Poisson with the best
\(R^2=0.5611089\) and the lowest \(RMSE=1.2757393\). We could go with
the basis of the AIC metric, however, our goal is to predict the best
results with our evaluation set. Even with all the transformations,
interactions and trying to handle zero inflation factors, a Poisson
model is our best choice to predict on the evaluation set.

Again AIC would be a better choice if we wanted to have more of an
inference between the models.

\begin{table}[H]
\centering\centering
\caption{\label{tab:model-results}Model Performance}
\centering
\begin{tabular}[t]{l|r|r|r|r|r}
\hline
  & Poisson & ZI\_Poisson & Neg\_Binom & ZI\_Neg\_Binom & Stepwise\\
\hline
RMSE & 1.2757393 & 1.2842321 & 1.2757404 & 1.2842601 & 1.2889856\\
\hline
Rsquared & 0.5611089 & 0.5554209 & 0.5611082 & 0.5554045 & 0.5519412\\
\hline
MAE & 1.0163856 & 0.9956513 & 1.0163873 & 0.9956453 & 1.0208232\\
\hline
AIC & 159979.7702125 & 145969.2867016 & 159983.1967454 & 145971.8130136 & 151064.1672811\\
\hline
\end{tabular}
\end{table}

\subsubsection{Predictions}\label{predictions}

\subsubsection{Predictions}\label{predictions-1}

As we have now selected our models, we are ready to make predictions on
the evaluation set. This is a slightly complicated process because our
second model is dependent on our first one. We complete this process
below:

\begin{table}[H]
\centering\centering
\caption{\label{tab:unnamed-chunk-33}Preview: Predictions for Evaluation Dataset}
\centering
\begin{tabular}[t]{r}
\hline
TARGET\\
\hline
3\\
\hline
4\\
\hline
2\\
\hline
2\\
\hline
2\\
\hline
5\\
\hline
3\\
\hline
5\\
\hline
1\\
\hline
3\\
\hline
\end{tabular}
\end{table}

\begin{center}\rule{0.5\linewidth}{0.5pt}\end{center}

\begin{verbatim}
## python:         C:/Users/shaya/AppData/Local/Programs/Python/Python312/python.exe
## libpython:      C:/Users/shaya/AppData/Local/Programs/Python/Python312/python312.dll
## pythonhome:     C:/Users/shaya/AppData/Local/Programs/Python/Python312
## version:        3.12.0 (tags/v3.12.0:0fb18b0, Oct  2 2023, 13:03:39) [MSC v.1935 64 bit (AMD64)]
## Architecture:   64bit
## numpy:          C:/Users/shaya/AppData/Local/Programs/Python/Python312/Lib/site-packages/numpy
## numpy_version:  1.26.1
## 
## NOTE: Python version was forced by use_python() function
\end{verbatim}

\begin{verbatim}
## Using virtual environment "~/.virtualenvs/r-reticulate" ...
\end{verbatim}

\begin{verbatim}
## Negative Binomial Regression Results:
\end{verbatim}

\begin{verbatim}
## MSE: 1.5850614200854611
\end{verbatim}

\begin{verbatim}
## MAE: 0.9922015145786195
\end{verbatim}

\begin{verbatim}
## <keras.src.callbacks.history.History object at 0x0000021A7F8AFEC0>
\end{verbatim}

\begin{verbatim}
## 
##   1/600 ━━━━━━━━━━━━━━━━━━━━ 29s 49ms/step
##  60/600 ━━━━━━━━━━━━━━━━━━━━ 0s 860us/step
## 133/600 ━━━━━━━━━━━━━━━━━━━━ 0s 768us/step
## 203/600 ━━━━━━━━━━━━━━━━━━━━ 0s 752us/step
## 276/600 ━━━━━━━━━━━━━━━━━━━━ 0s 735us/step
## 359/600 ━━━━━━━━━━━━━━━━━━━━ 0s 706us/step
## 426/600 ━━━━━━━━━━━━━━━━━━━━ 0s 714us/step
## 493/600 ━━━━━━━━━━━━━━━━━━━━ 0s 718us/step
## 563/600 ━━━━━━━━━━━━━━━━━━━━ 0s 718us/step
## 600/600 ━━━━━━━━━━━━━━━━━━━━ 0s 750us/step
\end{verbatim}

\begin{verbatim}
## 
## Neural Network Results:
\end{verbatim}

\begin{verbatim}
## MSE: 1.612623625917454
\end{verbatim}

\begin{verbatim}
## MAE: 0.9611635377341192
\end{verbatim}

\begin{verbatim}
## 
## Model Comparison:
\end{verbatim}

\begin{verbatim}
##                           Model       MSE       MAE
## 0  Negative Binomial Regression  1.585061  0.992202
## 1                Neural Network  1.612624  0.961164
\end{verbatim}

\begin{verbatim}
## (array([0, 1]), [Text(0, 0, 'Negative Binomial Regression'), Text(1, 0, 'Neural Network')])
\end{verbatim}

\includegraphics{test_files/figure-latex/unnamed-chunk-35-1.pdf}

\section{Conclusion:}\label{conclusion}

In this project, we aimed to develop predictive models for wine sales
using statistical techniques and machine learning algorithms. We started
by exploring the data, handling missing values, and transforming
variables. Initially, we experimented with Poisson regression and
zero-inflated Poisson models, as well as negative binomial regression to
address over-dispersion. However, the simple Poisson model emerged as
the best performer. We refined our models through stepwise regression
but found no significant improvement over the simple Poisson model.
Overall, our models offer valuable insights for wine producers and
distributors, aiding in resource allocation and marketing strategy
adjustments to optimize sales and enhance profitability in the wine
industry.

\begin{center}\rule{0.5\linewidth}{0.5pt}\end{center}

\begin{verbatim}
## python:         C:/Users/shaya/AppData/Local/Programs/Python/Python312/python.exe
## libpython:      C:/Users/shaya/AppData/Local/Programs/Python/Python312/python312.dll
## pythonhome:     C:/Users/shaya/AppData/Local/Programs/Python/Python312
## version:        3.12.0 (tags/v3.12.0:0fb18b0, Oct  2 2023, 13:03:39) [MSC v.1935 64 bit (AMD64)]
## Architecture:   64bit
## numpy:          C:/Users/shaya/AppData/Local/Programs/Python/Python312/Lib/site-packages/numpy
## numpy_version:  1.26.1
## 
## NOTE: Python version was forced by use_python() function
\end{verbatim}

\begin{verbatim}
## Using virtual environment "~/.virtualenvs/r-reticulate" ...
\end{verbatim}

\begin{verbatim}
## Negative Binomial Regression Results:
\end{verbatim}

\begin{verbatim}
## MSE: 1.5850614200854611
\end{verbatim}

\begin{verbatim}
## MAE: 0.9922015145786195
\end{verbatim}

\begin{verbatim}
## <keras.src.callbacks.history.History object at 0x00000219A4F6B350>
\end{verbatim}

\begin{verbatim}
## 
##   1/600 ━━━━━━━━━━━━━━━━━━━━ 29s 49ms/step
##  54/600 ━━━━━━━━━━━━━━━━━━━━ 0s 945us/step
## 119/600 ━━━━━━━━━━━━━━━━━━━━ 0s 856us/step
## 183/600 ━━━━━━━━━━━━━━━━━━━━ 0s 832us/step
## 251/600 ━━━━━━━━━━━━━━━━━━━━ 0s 806us/step
## 314/600 ━━━━━━━━━━━━━━━━━━━━ 0s 804us/step
## 378/600 ━━━━━━━━━━━━━━━━━━━━ 0s 802us/step
## 447/600 ━━━━━━━━━━━━━━━━━━━━ 0s 792us/step
## 515/600 ━━━━━━━━━━━━━━━━━━━━ 0s 787us/step
## 581/600 ━━━━━━━━━━━━━━━━━━━━ 0s 785us/step
## 600/600 ━━━━━━━━━━━━━━━━━━━━ 1s 822us/step
\end{verbatim}

\begin{verbatim}
## 
## Neural Network Results:
\end{verbatim}

\begin{verbatim}
## MSE: 1.6384511095542058
\end{verbatim}

\begin{verbatim}
## MAE: 0.9857123492540047
\end{verbatim}

\begin{verbatim}
## 
## Model Comparison:
\end{verbatim}

\begin{verbatim}
##                           Model       MSE       MAE
## 0  Negative Binomial Regression  1.585061  0.992202
## 1                Neural Network  1.638451  0.985712
\end{verbatim}

\begin{verbatim}
## (array([0, 1]), [Text(0, 0, 'Negative Binomial Regression'), Text(1, 0, 'Neural Network')])
\end{verbatim}

\includegraphics{test_files/figure-latex/unnamed-chunk-38-1.pdf}

\end{document}
